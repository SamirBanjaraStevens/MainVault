\documentclass[amssymb,twocolumn,pra,10pt,aps]{revtex4-1}
\usepackage{mathptmx,amsmath}

\begin{document}
\title{Solutions to the 58th William Lowell Putnam Mathematical Competition \\
    Saturday, December 6, 1997}
\author{Manjul Bhargava, Kiran Kedlaya, and Lenny Ng}
\noaffiliation
\maketitle

\begin{itemize}
\item[A--1]
The centroid $G$ of the triangle is collinear with $H$ and $O$
(Euler line), and the centroid lies two-thirds of the way from $A$ to
$M$. Therefore $H$ is also two-thirds of the way from $A$ to $F$, so
$AF = 15$. Since the triangles $BFH$ and $AFC$ are similar (they're
right triangles and
\[
\angle HBC = \pi/2 - \angle C = \angle CAF),
\]
we have
\[
BF/FH = AF/FC
\]
or
\[
BF \cdot FC = FH \cdot AF = 75.
\]
Now
\[
BC^2 = (BF + FC)^2 = (BF - FC)^2 + 4 BF \cdot FC,
\]
but
\[
BF - FC = BM+MF-(MC-MF) = 2MF = 22,
\]
so
\[
BC = \sqrt{22^2 + 4 \cdot 75} = \sqrt{784} = 28.
\]

\item[A--2]
We show more precisely that the game terminates with one player
holding all of the pennies if and only if $n = 2^m + 1$ or $n = 2^m +
2$ for some $m$. First suppose we are in the following situation for
some $k \geq 2$. (Note: for us, a ``move'' consists of two turns,
starting with a one-penny pass.)
\begin{itemize}
\item
Except for the player to move, each player has $k$ pennies;
\item
The player to move has at least $k$ pennies.
\end{itemize}
We claim then that
the game terminates if and only if the number of players is a
power of 2. First suppose the number of players is even; then after
$m$ complete rounds, every other player, starting with the player who
moved first, will have $m$ more pennies than initially, and the others
will all have 0. Thus we are reduced to the situation with half as
many players; by this process, we eventually reduce to the case where
the number of players is odd. However, if there is more than one
player, after two complete rounds everyone has as many pennies as they
did before (here we need $m \geq 2$), so the game fails to terminate.
This verifies the claim.

Returning to the original game, note that after one complete round,
$\lfloor \frac{n-1}{2} \rfloor$ players remain, each with 2 pennies
except for the player to move, who has
either 3 or 4 pennies. Thus by the above argument, the game terminates
if and only if $\lfloor \frac{n-1}{2} \rfloor$ is a power of 2, that
is, if and only if $n = 2^m + 1$ or $n = 2^m + 2$ for some $m$.

\item[A--3]
Note that the series on the left is simply $x \exp (-x^2/2)$.  By
integration by parts,
\[
\int_0^\infty x^{2n+1} e^{-x^2/2} dx =
2n \int_0^\infty x^{2n-1} e^{-x^2/2} dx
\]
and so by induction,
\[
\int_0^\infty x^{2n+1} e^{-x^2/2} dx =
2 \times 4 \times \cdots \times 2n.
\]
Thus the desired
integral is simply
\[
\sum_{n=0}^\infty \frac{1}{2^n n!} = \sqrt{e}.
\]

\item[A--4]
In order to have $\psi(x) = a \phi(x)$ for all $x$, we must in
particular have this for $x = e$, and so we take $a = \phi(e)^{-1}$.
We first note that
\[
\phi(g) \phi(e) \phi(g^{-1}) = \phi(e) \phi(g) \phi(g^{-1})
\]
and so $\phi(g)$ commutes with $\phi(e)$ for all $g$. Next, we note that
\[
\phi(x) \phi(y) \phi(y^{-1}x^{-1}) = \phi(e) \phi(xy) \phi(y^{-1}x^{-1})
\]
and using the commutativity of $\phi(e)$, we deduce
\[
\phi(e)^{-1} \phi(x) \phi(e)^{-1} \phi(y) = \phi(e)^{-1} \phi(xy)
\]
or $\psi(xy) = \psi(x) \psi(y)$, as desired.

\item[A--5]
We may discard any solutions for which $a_1 \neq a_2$, since those come in
pairs; so assume $a_1 = a_2$.  Similarly, we may assume that $a_3 = a_4$,
$a_5 = a_6$, $a_7 = a_8$, $a_9=a_{10}$.  Thus we get the equation
\[
2/a_1 + 2/a_3 + 2/a_5 + 2/a_7 + 2/a_9 = 1.
\]
Again, we may assume $a_1 = a_3$ and $a_5 = a_7$, so we get $4/a_1 + 4/a_5
+ 2/a_9 = 1$; and $a_1 = a_5$, so $8/a_1 + 2/a_9 = 1$.  This implies that
$(a_1-8)(a_9-2) = 16$, which by counting has 5 solutions.  Thus $N_{10}$
is odd.

\item[A--6]
Clearly $x_{n+1}$ is a polynomial in $c$ of degree $n$,
so it suffices to identify $n$ values of $c$ for which $x_{n+1} =
0$. We claim these are $c = n-1-2r$ for $r=0,1,\dots, n-1$; in this
case, $x_k$ is the coefficient of $t^{k-1}$ in the polynomial
$f(t) = (1-t)^r (1+t)^{n-1-r}$. This can be verified by noticing that
$f$ satisfies the differential equation
\[
\frac{f'(t)}{f(t)} = \frac{n-1-r}{1+t} - \frac{r}{1-t}
\]
(by logarithmic differentiation) or equivalently,
\begin{align*}
(1-t^2) f'(t) &= f(t) [(n-1-r)(1-t) - r(1+t)] \\
&= f(t) [(n-1-2r)  - (n-1)t]
\end{align*}
and then taking the coefficient of $t^{k}$ on both sides:
\begin{gather*}
(k+1) x_{k+2} - (k-1) x_k = \\
(n-1-2r) x_{k+1} - (n-1) x_{k}.
\end{gather*}
In particular, the largest such $c$ is $n-1$, and $x_k =
\binom{n-1}{k-1}$ for $k= 1, 2, \dots, n$.

Greg Kuperberg has suggested an alternate approach to show directly
that $c=n-1$ is the largest root, without computing the others. Note
that the condition $x_{n+1} = 0$ states that $(x_1, \dots, x_n)$ is an
eigenvector of the matrix
\[
A_{ij} = \left\{ \begin{array}{cc} i & j = i + 1 \\ n-j & j=i-1 \\
0&\mbox{otherwise} \end{array} \right.
\]
with eigenvalue $c$. By the Perron-Frobenius theorem, $A$ has a unique
eigenvector with positive entries, whose eigenvalue has modulus
greater than or equal to that of any other eigenvalue, which proves
the claim.

\item[B--1]
It is trivial to check that $\frac{m}{6n}=\{\frac{m}{6n}\}\leq
\{\frac{m}{3n}\}$ for $1\leq m\leq 2n$, that
$1-\frac{m}{3n}=\{\frac{m}{3n}\}\leq \{\frac{m}{6n}\}$ for $2n\leq
m\leq 3n$, that $\frac{m}{3n}-1=\{\frac{m}{3n}\}\leq \{\frac{m}{6n}\}$
for $3n\leq m\leq 4n$, and that $1-\frac{m}{6n}=\{\frac{m}{6n}\}\leq
\{\frac{m}{3n}\}$ for $4n\leq m\leq 6n$.  Therefore the desired sum is
\begin{gather*}
\sum_{m=1}^{2n-1} \frac{m}{6n}
 +\sum_{m=2n}^{3n-1} \left(1-\frac{m}{3n} \right) \\
 +\sum_{m=3n}^{4n-1} \left(\frac{m}{3n}-1 \right) + \sum_{m=4n}^{6n-1} \left(
1-\frac{m}{6n} \right)
=n.
\end{gather*}

\item[B--2]
It suffices to show that $|f(x)|$ is bounded for $x \geq 0$, since $f(-x)$
satisfies the same equation as $f(x)$.  But then
\begin{align*}
\frac{d}{dx}\left(
(f(x))^2 + (f'(x))^2 \right) &= 2f'(x)(f(x)+f''(x)) \\
&= -2xg(x)(f'(x))^2 \leq 0,
\end{align*}
so that $(f(x))^2 \leq (f(0))^2 + (f'(0))^2$ for $x\geq 0$.

\item[B--3]
The only such $n$ are the numbers 1--4, 20--24, 100--104, and
120--124.  For the proof let
\[H_n=\sum_{m=1}^n \frac{1}{m}\]
and introduce the auxiliary function
\[I_n=\sum_{1\leq m\leq n, (m,5)=1} \frac{1}{m}.\]
It is immediate (e.g., by induction) that
$I_n\equiv 1,-1,1,0,0$ (mod $5$) for $n\equiv  1,2,3,4,5$ (mod 5)
respectively, and moreover, we have the equality
\[\label{(*)}
H_n= \sum_{m=0}^k \frac{1}{5^m} I_{\lfloor n/5^m \rfloor},\]
where $k=k(n)$ denotes the largest integer such that $5^k\leq n$.
We wish to determine those $n$ such that the above sum has nonnegative
5--valuation.  (By the 5--valuation of a number $a$ we mean
the largest integer $v$ such that $a/5^v$ is an integer.)

If $\lfloor n/5^k \rfloor\leq 3$, then the last term in the above sum
has 5--valuation $-k$, since $I_1$, $I_2$, $I_3$ each have valuation
0; on the other hand, all other terms must have 5--valuation strictly
larger than $-k$.  It follows that $H_n$ has 5--valuation exactly
$-k$; in particular, $H_n$ has nonnegative 5--valuation in this case
if and only if $k=0$, i.e., $n=1$, 2, or 3.

Suppose now that $\lfloor n/5^k \rfloor=4$.  Then we must also have
$20\leq \lfloor n/5^{k-1}\rfloor \leq 24$.  The former condition
implies that the last term of the above sum is $I_4/5^k=1/(12\cdot
5^{k-2})$, which has 5--valuation $-(k-2)$.

It is clear that $I_{20}\equiv I_{24}\equiv 0$ (mod 25); hence if $\lfloor
n/5^{k-1}\rfloor$ equals 20 or 24, then the second--to--last term of the
above sum (if it exists) has valuation at least $-(k-3)$.  The
third--to--last term (if it exists) is of the form $I_r/5^{k-2}$, so that
the sum of the last term and the third to last term takes the form
$(I_r+1/12)/5^{k-2}$.  Since $I_r$ can be congruent only to 0,1, or -1
(mod 5), and $1/12\equiv 3$ (mod 5), we conclude that the sum of the
last term and third--to--last term has valuation $-(k-2)$, while all other
terms have valuation strictly higher.  Hence $H_n$ has nonnegative
5--valuation in this case only when $k\leq 2$, leading to the values
$n=4$ (arising from $k=0$), 20,24 (arising from $k=1$ and $\lfloor
n/5^{k-1}\rfloor = 20$ and 24 resp.), 101, 102, 103, and 104 (arising
from $k=2$, $\lfloor n/5^{k-1}\rfloor = 20$) and 120, 121, 122, 123,
and 124 (arising from $k=2$, $\lfloor n/5^{k-1}\rfloor=24$).

Finally, suppose $\lfloor n/5^k \rfloor=4$ and $\lfloor n/5^{k-1}
\rfloor=21$, 22, or 23.  Then as before, the first condition
implies that the last term of the sum in (*) has valuation $-(k-2)$,
while the second condition implies that the second--to--last term in the
same sum has valuation $-(k-1)$.  Hence all terms in the sum (*) have
5--valuation strictly higher than $-(k-1)$, except for the
second--to--last term, and therefore $H_n$ has 5--valuation $-(k-1)$ in
this case.  In particular, $H_n$ is integral (mod 5) in this case if and
only if  $k\leq 1$, which gives the additional values $n=21$, 22, and 23.

\item[B--4]
Let $s_k = \sum_i (-1)^{i} a_{k-1,i}$ be the given sum (note that
$a_{k-1,i}$ is nonzero precisely for $i = 0, \dots, \lfloor
\frac{2k}{3} \rfloor)$. Since
\[
a_{m+1,n} = a_{m,n} + a_{m,n-1} + a_{m,n-2},
\]
we have
\begin{align*}
s_k - s_{k-1} + s_{k+2}
&= \sum_i (-1)^i (a_{n-i,i} + a_{n-i,i+1} + a_{n-i,i+2}) \\
&= \sum_i (-1)^i a_{n-i+1,i+2} = s_{k+3}.
\end{align*}
By computing $s_0 = 1, s_1 = 1, s_2 = 0$, we may easily verify by
induction that $s_{4j} = s_{4j+1} = 1$ and $s_{4j+2} = s_{4j+3} = 0$
for all $j \geq 0$. (Alternate solution suggested by John Rickert:
write $S(x,y) = \sum_{i=0}^\infty (y+xy^2+x^2y^3)^i$, and note
note that $s_k$ is the coefficient of $y^k$ in $S(-1,y) = (1+y)/(1-y^4)$.)

\item[B--5]
Define the sequence $x_1 = 2$, $x_n = 2^{x_{n-1}}$ for $n > 1$. It
suffices to show that for every $n$, $x_m \equiv x_{m+1} \equiv \cdots
\pmod n$ for some $m < n$. We do this by induction on $n$, with $n=2$
being obvious.

Write $n = 2^a b$, where $b$ is odd. It suffices to show that $x_m
\equiv \cdots$ modulo $2^a$ and modulo $b$, for some $m < n$. For the
former, we only need $x_{n-1} \geq a$, but clearly
$x_{n-1} \geq n$ by induction on $n$. For the latter, note that
$x_m \equiv x_{m+1} \equiv \cdots
\pmod b$ as long as $x_{m-1} \equiv x_m \equiv \cdots \pmod{\phi(b)}$,
where $\phi(n)$ is the Euler totient function. By hypothesis, this
occurs for some $m < \phi(b) + 1 \leq n$. (Thanks to Anoop Kulkarni
for catching a lethal typo in an earlier version.)

\item[B--6]
The answer is $25/13$.  Place the triangle on the cartesian plane so
that its vertices are at $C=(0,0), A=(0,3), B=(4,0)$.  Define also the points 
$D=(20/13,24/13),$ and $E=(27/13,0)$. We then compute that
\begin{align*}
\frac{25}{13} &= AD=BE=DE\\
\frac{27}{13} &= BC - CE = BE < BC \\
\frac{39}{13} &= AC < \sqrt{AC^2 + CE^2} = AE \\
\frac{40}{13} &= AB - AD = BD < AB
\end{align*}
and that $AD < CD$. In any dissection of the triangle into four parts, some two of $A,B,C,D,E$ must belong to the same part, forcing the least diameter to be at least $25/13$.

We now exhibit a dissection with least diameter $25/13$. (Some
variations of this dissection are possible.) Put $F = (15/13, 19/13)$,
$G = (15/13, 0)$, $H = (0, 19/13)$, $J = (32/15, 15/13)$,
and divide $ABC$ into the convex polygonal regions $ADFH$, $BEJ$, $CGFH$,
$DFGEJ$. 
To check that this dissection has least diameter $25/13$, it suffices (by the following remark) to check that the distances
\begin{gather*}
AD, AF, AH, BE, BJ, DE, CF, CG, CH, \\
DF, DG, DH, DJ, EF, EG, EJ, FG, FH, FJ, GJ
\end{gather*}
are all at most $25/13$. This can be checked by a long numerical calculation, which we omit in favor of some shortcuts: note that $ADFH$ and $BEJ$ are contained in
circular sectors centered at $A$ and $B$, respectively, of radius
$25/13$ and angle less than $\pi/3$, while $CGFH$ is a rectangle with diameter $CF < 25/13$.

\noindent
\textbf{Remark.} The preceding argument uses implicitly the fact that for $P$ a simple closed polygon in the plane, if we let $S$ denote the set of points on or within $P$, then  the maximum distance between two points of $S$ occurs between some pair of vertices of $P$.
This is an immediate consequence of the compactness of $S$ (which guarantees the existence of a maximum) and the convexity of the function taking $(x,y) \in S \times S$ to the squared distance between $x$ and $y$ (which is obvious in terms of Cartesian coordinates).

\end{itemize}
\end{document}



