\documentclass[amssymb,twocolumn,pra,10pt,aps]{revtex4-1}
\usepackage{mathptmx,amsmath}

\newcommand{\RR}{\mathbb{R}}
\newcommand{\CC}{\mathbb{C}}
\newcommand{\QQ}{\mathbb{Q}}
\newcommand{\ZZ}{\mathbb{Z}}

\begin{document}
\title{Solutions to the 63rd William Lowell Putnam Mathematical Competition \\
    Saturday, December 7, 2002}
\author{Kiran Kedlaya and Lenny Ng}
\noaffiliation
\maketitle

\begin{itemize}

\item[A--1]
By differentiating $P_n(x)/(x^k-1)^{n+1}$, we find that
$P_{n+1}(x) = (x^k-1)P_n'(x)-(n+1)kx^{k-1}P_n(x)$; substituting
$x=1$ yields $P_{n+1}(1) = -(n+1)k P_n(1)$.  Since $P_0(1)=1$, an
easy induction gives $P_n(1) = (-k)^n n!$ for all $n \geq 0$.

Note: one can also argue by expanding in Taylor series around $1$.
Namely, we have
\[
\frac{1}{x^k - 1} = \frac{1}{k(x-1) + \cdots}
= \frac{1}{k} (x-1)^{-1} + \cdots,
\]
so
\[
\frac{d^n}{dx^n} \frac{1}{x^k - 1}
= \frac{(-1)^n n!}{k (x-1)^{-n-1}}
\]
and
\begin{align*}
P_n(x) &= (x^k - 1)^{n+1} \frac{d^n}{dx^n} \frac{1}{x^k - 1} \\
&= (k (x-1) + \cdots)^{n+1} \left( \frac{(-1)^n n!}{k}(x-1)^{-n-1}
+ \cdots \right) \\
&= (-k)^n n! + \cdots.
\end{align*}

\item[A--2]
Draw a great circle through two of the points. There are two
closed hemispheres with this great circle as boundary, and each of
the other three points lies in one of them. By the pigeonhole principle,
two of those three points lie in the same hemisphere, and that hemisphere
thus contains four of the five given points.

Note: by a similar argument, one can prove that among any $n+3$ points on
an $n$-dimensional sphere, some $n+2$ of them lie on a closed hemisphere.
(One cannot get by with only $n+2$ points: put them at the vertices of
a regular simplex.)
Namely, any $n$ of the points lie on a great sphere, which forms the boundary
of two hemispheres; of the remaining three points, some two lie in the
same hemisphere.

\item[A--3]
Note that each of the sets $\{1\}, \{2\}, \dots, \{n\}$ has the
desired property. Moreover, for each set $S$ with integer average $m$
that does not contain $m$, $S \cup \{m\}$ also has average $m$,
while for each set $T$ of more than one element with integer average
$m$ that contains $m$, $T \setminus \{m\}$ also has average $m$.
Thus the subsets other than $\{1\}, \{2\}, \dots, \{n\}$ can be grouped
in pairs, so $T_n - n$ is even.

\item[A--4]
(partly due to David Savitt)
Player 0 wins with optimal play. In fact, we prove that Player 1 cannot
prevent Player 0 from creating a row of all zeroes, a column of all
zeroes, or a $2 \times 2$ submatrix of all zeroes. Each of these forces
the determinant of the matrix to be zero.

For $i,j=1, 2,3$, let $A_{ij}$ denote the position in row $i$ and
column $j$. Without loss of generality, we may assume that Player
1's first move is at $A_{11}$. Player 0 then plays at $A_{22}$:
\[
\begin{pmatrix}
1 & * & * \\
* & 0 & * \\
* & * & *
\end{pmatrix}
\]
After Player 1's second move, at least one of $A_{23}$ and $A_{32}$
remains vacant. Without loss of generality, assume $A_{23}$ remains
vacant; Player 0 then plays there.

After Player 1's third move, Player 0 wins by playing at $A_{21}$ if that
position is unoccupied. So assume instead that Player 1 has played there.
Thus of Player 1's three moves so far, two are at $A_{11}$ and $A_{21}$.
Hence for $i$ equal to one of 1 or 3, and for $j$ equal to one of 2 or 3,
the following are both true:
\begin{enumerate}
\item[(a)]
The $2 \times 2$ submatrix formed by rows 2 and $i$ and by columns
2 and 3 contains two zeroes and two empty positions.
\item[(b)]
Column $j$ contains one zero and two empty positions.
\end{enumerate}
Player 0 next plays at $A_{ij}$. To prevent a zero column, Player 1
must play in column $j$, upon which Player 0 completes the $2 \times 2$
submatrix in (a) for the win.

Note: one can also solve this problem directly by making a tree of
possible play sequences. This tree can be considerably collapsed
using symmetries: the symmetry between rows and columns, the invariance
of the outcome under reordering of rows or columns, and the fact that
the scenario after a sequence of moves does not depend on the order of
the moves (sometimes called ``transposition invariance'').

Note (due to Paul Cheng):
one can reduce Determinant
Tic-Tac-Toe to a variant of ordinary tic-tac-toe.
Namely, consider a tic-tac-toe grid
labeled as follows:
\[
\begin{array}{c|c|c}
A_{11} & A_{22} & A_{33} \\
\hline
A_{23} & A_{31} & A_{12} \\
\hline
A_{32} & A_{13} & A_{21}
\end{array}
\]
Then each term in the expansion of the determinant occurs in a row
or column of the grid. Suppose Player 1 first plays in the top left.
Player 0 wins by playing first in the top row, and second in the left
column. Then there are only one row and column left for Player 1
to threaten, and Player 1 cannot already threaten both on the third move,
so Player 0 has time to block both.

\item[A--5]
It suffices to prove that for any relatively prime positive integers
$r,s$, there exists an integer $n$ with $a_n = r$ and $a_{n+1} = s$.
We prove this by induction on $r+s$, the case $r+s=2$ following
from the fact that $a_0=a_1 = 1$. Given $r$ and $s$ not both 1 with
$\gcd(r,s) = 1$, we must have $r \neq s$. If $r>s$, then by
the induction hypothesis we have $a_n = r-s$ and $a_{n+1} = s$ for
some $n$; then $a_{2n+2} = r$ and $a_{2n+3} = s$. If $r< s$,
then we have $a_n = r$ and $a_{n+1} = s-r$ for some $n$; then
$a_{2n+1} = r$ and $a_{2n+2} = s$.

Note: a related problem is as follows. Starting with the sequence
\[
\frac{0}{1}, \frac{1}{0},
\]
repeat the following operation: insert between each pair
$\frac{a}{b}$ and $\frac{c}{d}$ the pair $\frac{a+c}{b+d}$.
Prove that each positive rational number eventually appears.

Observe that by induction, if $\frac{a}{b}$ and $\frac{c}{d}$
are consecutive terms in the sequence, then $bc - ad = 1$. The
same holds for consecutive terms of the $n$-th \emph{Farey sequence}, the
sequence of rational numbers in $[0,1]$ with denominator
(in lowest terms) at most $n$.

\item[A--6]
The sum converges for $b=2$ and diverges for $b \geq 3$.
We first consider $b \geq 3$. Suppose the sum converges;
then the fact
that $f(n) = n f(d)$ whenever $b^{d-1} \leq n \leq b^{d} - 1$ yields
\begin{equation} \label{a6eq1}
\sum_{n=1}^\infty \frac{1}{f(n)}
= \sum_{d=1}^\infty \frac{1}{f(d)} \sum_{n=b^{d-1}}^{b^d - 1} \frac{1}{n}.
\end{equation}
However, by comparing the integral of $1/x$ with a Riemann sum,
we see that
\begin{align*}
\sum_{n=b^{d-1}}^{b^d - 1} \frac{1}{n}
&> \int_{b^{d-1}}^{b^d} \frac{dx}{x} \\
&= \log (b^d) - \log (b^{d-1}) = \log b,
\end{align*}
where $\log$ denotes the natural logarithm. Thus \eqref{a6eq1} yields
\[
\sum_{n=1}^\infty \frac{1}{f(n)}
> (\log b) \sum_{n=1}^\infty \frac{1}{f(n)},
\]
a contradiction since $\log b > 1$ for $b \geq 3$. Therefore the
sum diverges.

For $b=2$, we have a slightly different identity because $f(2) \neq
2 f(2)$. Instead, for any positive integer $i$, we have
\begin{equation} \label{a6eq2}
\sum_{n=1}^{2^i-1} \frac{1}{f(n)}
= 1 + \frac{1}{2} + \frac{1}{6} +
\sum_{d=3}^i \frac{1}{f(d)} \sum_{n=2^{d-1}}^{2^d - 1} \frac{1}{n}.
\end{equation}
Again comparing an integral to a Riemann sum, we see that for $d\geq 3$,
\begin{align*}
\sum_{n=2^{d-1}}^{2^d - 1} \frac{1}{n} &<
\frac{1}{2^{d-1}} - \frac{1}{2^d} + \int_{2^{d-1}}^{2^d} \frac{dx}{x}
\\
&= \frac{1}{2^d} + \log 2 \\
&\leq \frac{1}{8} + \log 2 < 0.125 + 0.7 < 1.
\end{align*}
Put $c = \frac{1}{8} + \log 2$ and $L = 1+\frac{1}{2} +
\frac{1}{6(1-c)}$. Then we can prove that $\sum_{n=1}^{2^i-1}
\frac{1}{f(n)} < L$ for all $i \geq 2$ by induction on $i$. The case
$i=2$ is clear. For the induction, note that by \eqref{a6eq2},
\begin{align*}
\sum_{n=1}^{2^i-1} \frac{1}{f(n)}
&< 1 + \frac{1}{2} + \frac{1}{6} + c \sum_{d=3}^i \frac{1}{f(d)} \\
&< 1 + \frac{1}{2} + \frac{1}{6} + c \frac{1}{6(1-c)} \\
&= 1 + \frac{1}{2} + \frac{1}{6(1-c)} = L,
\end{align*}
as desired. We conclude that $\sum_{n=1}^\infty \frac{1}{f(n)}$
converges to a limit less than or equal to $L$.

Note: the above argument proves that the sum for $b=2$ is at most
$L < 2.417$. One can also obtain a lower bound by the same technique,
namely $1 + \frac{1}{2} + \frac{1}{6(1 - c')}$ with $c' = \log 2$.
This bound exceeds $2.043$. (By contrast, summing the first 100000 terms of
the series only yields a lower bound of $1.906$.)
Repeating the same arguments with $d \geq 4$
as the cutoff yields the upper bound $2.185$ and the lower bound $2.079$.

\item[B--1]
The probability is $1/99$. In fact, we show by induction on $n$
that after $n$ shots, the probability of having made any number of
shots from $1$ to $n-1$ is equal to $1/(n-1)$. This is evident
for $n=2$. Given the result for $n$, we see that the probability of
making $i$ shots after $n+1$ attempts is
\begin{align*}
\frac{i-1}{n} \frac{1}{n-1} + \left( 1 - \frac{i}{n} \right) \frac{1}{n-1}
&= \frac{(i-1) + (n-i)}{n(n-1)} \\
&= \frac{1}{n},
\end{align*}
as claimed.

\item[B--2]
(Note: the problem statement assumes that all polyhedra are connected
and that no two edges share more than one face,
so we will do likewise. In particular, these are true for all convex
polyhedra.)
We show that in fact the first player can win on the third move.
Suppose the polyhedron has a face $A$ with at least four edges. If
the first player plays there first, after the second player's first move
there will be three consecutive faces $B,C,D$ adjacent to $A$ which
are all unoccupied. The first player wins by playing in $C$; after
the second player's second move, at least one of $B$ and $D$ remains
unoccupied, and either is a winning move for the first player.

It remains to show that the polyhedron has a face with at least four
edges. (Thanks to Russ Mann for suggesting the following argument.)
Suppose on the contrary that each face has only three edges.
Starting with any face $F_1$ with vertices $v_1, v_2, v_3$, let
$v_4$ be the other endpoint of the third edge out of $v_1$. Then
the faces adjacent to $F_1$ must have vertices $v_1, v_2, v_4$;
$v_1, v_3, v_4$; and $v_2, v_3, v_4$. Thus $v_1, v_2, v_3, v_4$ form
a polyhedron by themselves, contradicting the fact that the given
polyhedron is connected and has at least five vertices.
(One can also deduce this using Euler's formula
$V - E + F = 2 - 2g$, where $V,E,F$ are the numbers of vertices,
edges and faces, respectively, and $g$ is the genus of the polyhedron.
For a convex polyhedron, $g=0$ and you get the ``usual'' Euler's formula.)

Note: Walter Stromquist points out the following counterexample if
one relaxes the assumption that a pair of faces may not share multiple
edges. Take a tetrahedron and remove a smaller tetrahedron from the
center of an edge; this creates two small triangular faces and turns two
of the original faces into hexagons. Then the second player can draw
by signing one of the hexagons, one of the large triangles, and one
of the small triangles. (He does this by ``mirroring'': wherever the first
player signs, the second player signs the other face of the same type.)

\item[B--3]
The desired inequalities can be rewritten as
\[
1 - \frac{1}{n} < \exp\left( 1 + n \log \left( 1 - \frac{1}{n} \right)
\right) < 1 - \frac{1}{2n}.
\]
By taking logarithms, we can rewrite the desired inequalities as
\begin{align*}
-\log \left( 1 - \frac{1}{2n} \right)
&< -1 - n \log \left( 1 - \frac{1}{n} \right) \\
&< -\log \left( 1 - \frac{1}{n} \right).
\end{align*}
Rewriting these in terms of the Taylor expansion of
$-\log(1-x)$, we see that the desired result is also equivalent
to
\[
\sum_{i=1}^\infty \frac{1}{i 2^i n^i}
< \sum_{i=1}^\infty \frac{1}{(i+1) n^i}
< \sum_{i=1}^\infty \frac{1}{i n^i},
\]
which is evident because the inequalities hold term by term.

Note: David Savitt points out that the upper bound can be improved from
$1/(ne)$ to $2/(3ne)$ with a slightly more complicated argument. (In
fact, for any $c>1/2$, one has an upper bound of $c/(ne)$, but only
for $n$ above a certain bound depending on $c$.)

\item[B--4]
Use the following strategy: guess $1, 3, 4, 6, 7, 9, \dots$
until the target number $n$ is revealed to be equal to or lower than one
of these guesses. If $n \equiv 1 \pmod{3}$, it will be guessed on an
odd turn. If $n \equiv 0 \pmod{3}$, it will be guessed on an even turn.
If $n \equiv 2 \pmod{3}$, then $n+1$ will be guessed on an even turn,
forcing a guess of $n$ on the next turn. Thus the probability
of success with this strategy is $1335/2002 > 2/3$.

Note: for any positive integer $m$, this strategy wins when the
number is being guessed from $[1,m]$ with probability
$\frac{1}{m} \lfloor \frac{2m+1}{3} \rfloor$. We can prove that
this is best possible as follows.
Let $a_m$ denote $m$ times
the probability of winning when playing optimally.  Also, let $b_m$
denote $m$ times the corresponding probability of winning if the
objective is to select the number in an even number of guesses
instead.  (For definiteness, extend the definitions to incorporate
$a_0 = 0$ and $b_0=0$.)

We first claim that $a_m = 1 + \max_{1\leq k\leq m} \{b_{k-1} +
b_{m-k}\}$ and $b_m = \max_{1\leq k\leq m} \{a_{k-1} + a_{m-k}\}$ for $m
\geq 1$.  To establish the first recursive identity, suppose that our
first guess is some integer $k$. We automatically win if $n=k$, with
probability $1/m$. If $n<k$, with probability $(k-1)/m$, then we wish
to guess an integer in $[1,k-1]$ in an even number of guesses; the
probability of success when playing optimally is $b_{k-1}/(k-1)$, by
assumption.  Similarly, if $n<k$, with probability $(m-k)/m$, then the
subsequent probability of winning is $b_{m-k}/(m-k)$.  In sum, the
overall probability of winning if $k$ is our first guess is
$(1+b_{k-1}+b_{m-k})/m$. For optimal strategy, we choose $k$ such that
this quantity is maximized. (Note that this argument still holds if
$k=1$ or $k=m$, by our definitions of $a_0$ and $b_0$.)  The first
recursion follows, and the second recursion is established similarly.

We now prove by induction that $a_m = \lfloor (2m+1)/3 \rfloor$ and
$b_m = \lfloor 2m/3 \rfloor$ for $m \geq 0$. The inductive step relies
on the inequality $\lfloor x \rfloor + \lfloor y \rfloor \leq \lfloor
x+y \rfloor$, with equality when one of $x,y$ is an integer. Now
suppose that $a_i = \lfloor (2i+1)/3 \rfloor$ and
$b_i = \lfloor 2i/3 \rfloor$ for $i < m$. Then
\begin{align*}
1+b_{k-1}+b_{m-k} &= 1+\left\lfloor \frac{2(k-1)}{3} \right\rfloor +
\left\lfloor
\frac{2(m-k)}{3} \right\rfloor \\
&\leq \left\lfloor \frac{2m}{3} \right\rfloor
\end{align*}
and similarly $a_{k-1}+a_{m-k} \leq \lfloor (2m+1)/3 \rfloor$, with
equality in both cases attained, e.g., when $k=1$.
The inductive formula for $a_m$ and $b_m$ follows.

\item[B--5]
(due to Dan Bernstein)
Put $N = 2002!$. Then for $d=1, \dots, 2002$,
the number $N^2$ written in base $b = N/d - 1$ has digits
$d^2, 2d^2, d^2$. (Note that these really are digits because
$2(2002)^2 < (2002!)^2/2002 - 1$.)

Note: one can also produce an integer $N$ which has base $b$
digits $1, *, 1$ for $n$ different values of $b$, as follows.
Choose $c$ with $0 < c < 2^{1/n}$. For $m$ a large positive integer,
put $N = 1 + (m+1)\cdots (m+n)\lfloor cm \rfloor^{n-2}$.
For $m$ sufficiently large, the bases
\[
b = \frac{N-1}{(m+i)m^{n-2}} = \prod_{j \neq i} (m+j)
\]
for $i=1, \dots, n$ will
have the properties that $N \equiv 1 \pmod{b}$ and $b^2 < N < 2b^2$
for $m$ sufficiently large.

Note (due to Russ Mann):
one can also give a ``nonconstructive'' argument. Let $N$ be a
large positive integer. For $b \in (N^2, N^3)$, the number of 3-digit
base-$b$ palindromes in the range $[b^2, N^6 - 1]$ is at least
\[
\left\lfloor \frac{N^6 - b^2}{b} \right\rfloor - 1
\geq \frac{N^6}{b^2} - b - 2,
\]
since there is a palindrome in each interval $[kb, (k+1)b-1]$ for
$k=b, \dots, b^2-1$. Thus the average number of bases for which
a number in $[1, N^6-1]$ is at least
\[
\frac{1}{N^6} \sum_{b=N^2+1}^{N^3-1} \left( \frac{N^6}{b} - b-2 \right)
\geq  \log(N) - c
\]
for some constant $c>0$. Take $N$ so that the right side exceeds $2002$;
then at least one number in $[1, N^6-1]$ is a base-$b$ palindrome
for at least 2002 values of $b$.

\item[B--6]
We prove that the determinant is congruent modulo $p$ to
\begin{equation} \label{b6eq2}
x \prod_{i=0}^{p-1} (y+ix) \prod_{i,j=0}^{p-1} (z + ix + jy).
\end{equation}
We first check that
\begin{equation} \label{b6eq1}
\prod_{i=0}^{p-1} (y+ix) \equiv y^p - x^{p-1} y \pmod{p}.
\end{equation}
Since both sides are homogeneous as polynomials in $x$ and $y$,
it suffices to check \eqref{b6eq1}
for $x=1$, as a congruence between polynomials. Now note that
the right side has $0,1,\dots,p-1$ as roots modulo $p$, as does
the left side. Moreover, both sides have the same leading coefficient.
Since they both have degree only $p$, they must then coincide.

We thus have
\begin{multline*}
x \prod_{i=0}^{p-1} (y+ix) \prod_{i,j=0}^{p-1} (z + ix + jy) \\
\begin{aligned}
&\equiv x (y^p - x^{p-1}y) \prod_{j=0}^{p-1} ((z+jy)^p - x^{p-1} (z+jy)) \\
&\equiv (xy^p - x^p y) \prod_{j=0}^{p-1} (z^p - x^{p-1} z + j y^p - j x^{p-1}
y) \\
&\equiv (xy^p - x^p y) ((z^p - x^{p-1} z)^p \\
&\quad - (y^p - x^{p-1}y)^{p-1}(z^p
- x^{p-1}z)) \\
 &\equiv (xy^p - x^p y)(z^{p^2} - x^{p^2 - p}z^p) \\
&\quad - x(y^p - x^{p-1}y)^p
(z^p - x^{p-1}z) \\
&\equiv xy^p z^{p^2} - x^p y z^{p^2} - x^{p^2-p+1} y^p z^p
+ x^{p^2} y z^p \\
&\quad - xy^{p^2}z^p + x^{p^2-p+1} y^p z^p
+ x^py^{p^2}z - x^{p^2} y^p z \\
&\equiv x y^p z^{p^2} + y z^p x^{p^2} + z x^p y^{p^2} \\
&\quad
- x z^p y^{p^2} - y x^p z^{p^2} - z y^p x^{p^2},
\end{aligned}
\end{multline*}
which is precisely the desired determinant.

Note: a simpler conceptual proof is as follows. (Everything in this
paragraph will be modulo $p$.) Note that for any
integers $a,b,c$, the column vector $[ax + by + cz, (ax + by + cz)^p,
(ax + by + cz)^{p^2}]$ is a linear combination of the columns of the
given matrix. Thus $ax+by+cz$ divides the determinant.
In particular, all of the factors of \eqref{b6eq2} divide the determinant;
since both \eqref{b6eq2} and the determinant have degree $p^2+p+1$,
they agree up to a scalar multiple. Moreover, they have the same
coefficient of $z^{p^2} y^p x$ (since this term only appears in
the expansion of \eqref{b6eq2} when you choose the first term in
each factor). Thus the determinant is congruent to \eqref{b6eq2}, as desired.

Either argument can be used to generalize to a corresponding $n \times n$
determinant, called a Moore determinant;
we leave the precise formulation to the reader. Note
the similarity with the classical Vandermonde determinant: if
$A$ is the $n \times n$ matrix with $A_{ij} = x_i^j$ for
$i,j=0, \dots, n-1$, then
\[
\det(A) = \prod_{1 \leq i < j \leq n} (x_j - x_i).
\]

\end{itemize}

\end{document}



