\documentclass[amssymb,twocolumn,pra,10pt,aps]{revtex4-1}
\usepackage{mathptmx,amsmath}

\newcommand{\RR}{\mathbb{R}}
\newcommand{\CC}{\mathbb{C}}
\DeclareMathOperator{\lcm}{lcm}
\DeclareMathOperator{\sgn}{sgn}

\begin{document}
\title{Solutions to the 67th William Lowell Putnam Mathematical Competition \\
    Saturday, December 2, 2006}
\author{Kiran Kedlaya and Lenny Ng}
\noaffiliation
\maketitle

\begin{itemize}

\item[A--1]
We change to cylindrical coordinates, i.e., we put $r = \sqrt{x^2 + y^2}$.
Then the given inequality is equivalent to
\[
r^2 + z^2 + 8 \leq 6r,
\]
or
\[
(r-3)^2 + z^2 \leq 1.
\]
This defines a solid of revolution (a solid torus); the area being rotated
is the disc $(x-3)^2 + z^2 \leq 1$ in the $xz$-plane. By Pappus's theorem,
the volume of this equals the area of this disc, which is $\pi$, times the
distance through which the center of mass is being rotated, which is $(2\pi)3$.
That is, the total volume is $6 \pi^2$.

\item[A--2]
Suppose on the contrary that the set $B$ of values of $n$ for which Bob
has a winning strategy is finite; for convenience, we include $n=0$ in $B$,
and write $B = \{b_1, \dots, b_m\}$.
Then for every nonnegative integer $n$ not
in $B$, Alice must have some move on a heap of $n$ stones leading to a
position in which the second player wins. That is, every nonnegative integer
not in $B$ can be written as $b + p - 1$ for some $b \in B$ and some prime
$p$. However, there are numerous ways to show that this cannot happen.

\textbf{First solution:}
Let $t$ be any integer bigger than all of the $b \in B$. Then it is easy to
write down $t$ consecutive composite integers, e.g., $(t+1)! + 2, \dots,
(t+1)! + t+1$. Take $n = (t+1)! + t$; then for each $b \in B$,
$n - b + 1$ is one of the composite integers we just wrote down.

\textbf{Second solution:}
Let $p_1, \dots, p_{2m}$ be
any prime numbers; then by the Chinese remainder theorem, there exists a
positive integer $x$ such that
\begin{align*}
x - b_1 &\equiv -1 \pmod{p_1 p_{m+1}} \\
\dots \\
x - b_n &\equiv -1 \pmod{p_m p_{2m}}.
\end{align*}
For each $b \in B$,
the unique integer $p$ such that $x=b+p-1$ is divisible
by at least two primes, and so cannot itself be prime.

\textbf{Third solution:} (by Catalin Zara)
Put $b_1 = 0$, and take $n = (b_2 - 1)\cdots(b_m - 1)$; then $n$ is
composite because $3, 8 \in B$, and for any nonzero $b \in B$,
$n - b_i + 1$ is divisible by but not equal to $b_i - 1$.
(One could also take $n = b_2 \cdots b_m - 1$, so that
$n-b_i+1$ is divisible by $b_i$.)

\item[A--3]
We first observe that given any sequence of integers
$x_1, x_2, \dots$ satisfying a recursion
\[
x_k = f(x_{k-1}, \dots, x_{k-n}) \qquad (k > n),
\]
where $n$ is fixed and $f$ is a fixed polynomial of $n$ variables with
integer coefficients, for any positive integer $N$, the sequence modulo $N$
is eventually periodic. This is simply because there are only finitely many
possible sequences of $n$ consecutive values modulo $N$, and once such
a sequence is repeated, every subsequent value is repeated as well.

We next observe that if one can rewrite the same recursion as
\[
x_{k-n} = g(x_{k-n+1}, \dots, x_k) \qquad (k > n),
\]
where $g$ is also a polynomial with integer coefficients, then
the sequence extends uniquely to a doubly infinite sequence $\dots,
x_{-1}, x_0, x_1, \dots$ which is fully periodic modulo any $N$.
That is the case in the
situation at hand, because we can rewrite the given recursion as
\[
x_{k-2005} = x_{k+1} - x_k.
\]
It thus suffices to find 2005 consecutive terms divisible by $N$ in the
doubly infinite sequence, for any fixed $N$ (so in particular for
$N = 2006$).
Running the recursion backwards, we easily find
\begin{gather*}
x_1 = x_0 = \cdots = x_{-2004} = 1 \\
x_{-2005} = \cdots = x_{-4009} = 0,
\end{gather*}
yielding the desired result.

\item[A--4]
\textbf{First solution:}
By the linearity of expectation, the average number of local maxima is equal
to the sum of the probability of having a local maximum at $k$ over
$k=1,\dots, n$.
For $k=1$, this probability is 1/2: given the pair
$\{\pi(1), \pi(2)\}$, it is equally likely that $\pi(1)$ or $\pi(2)$ is
bigger. Similarly, for $k=n$, the probability is 1/2. For $1 < k < n$,
the probability is 1/3: given the pair $\{\pi(k-1), \pi(k), \pi(k+1)\}$,
it is equally likely that any of the three is the largest.
Thus the average number of local maxima is
\[
2 \cdot \frac{1}{2} + (n-2) \cdot \frac{1}{3} =
\frac{n+1}{3}.
\]

\textbf{Second solution:}
Another way to apply the linearity of expectation is to compute the
probability that $i \in \{1, \dots, n\}$ occurs as a local maximum.
The most efficient way to do this is to imagine the permutation
as consisting of the symbols $1, \dots, n, *$ written in a circle in
some order. The number $i$ occurs as a local maximum if the two symbols
it is adjacent to both belong to the set $\{*, 1, \dots, i-1\}$. There are
$i(i-1)$ pairs of such symbols and $n(n-1)$ pairs in total, so the
probability of $i$ occurring as a local maximum is $i(i-1)/(n(n-1))$, and
the average number of local maxima is
\begin{align*}
\sum_{i=1}^n \frac{i(i-1)}{n(n-1)} &=
\frac{2}{n(n-1)} \sum_{i=1}^n \binom{i}{2} \\
&= \frac{2}{n(n-1)} \binom{n+1}{3} \\
&= \frac{n+1}{3}.
\end{align*}
One can obtain a similar (if slightly more intricate)
solution inductively, by removing the known
local maximum $n$ and splitting into two shorter sequences.

\textbf{Remark:}
The usual term for a local maximum in this sense is a \emph{peak}.
The complete distribution for the number of peaks is known;
Richard Stanley suggests the reference:
F. N. David and D. E. Barton, \textit{Combinatorial Chance}, Hafner, New York,
1962, p.\ 162 and subsequent.

\item[A--5]
Since the desired expression involves symmetric functions of $a_1,
\dots, a_n$, we start by finding a polynomial with $a_1, \dots, a_n$
as roots. Note that
\[
1 \pm i \tan \theta = e^{\pm i \theta} \sec \theta
\]
so that
\[
1 + i \tan \theta = e^{2 i \theta} (1 - i \tan \theta).
\]
Consequently, if we put $\omega = e^{2 i n \theta}$, then the polynomial
\[
Q_n(x) = (1 + ix)^n - \omega (1-ix)^n
\]
has among its roots $a_1, \dots, a_n$. Since these are distinct and
$Q_n$ has degree $n$, these must be exactly the roots.

If we write
\[
Q_n(x) = c_n x^n + \cdots + c_1 x + c_0,
\]
then $a_1 + \cdots + a_n = -c_{n-1}/c_n$ and $a_1\cdots a_n = -c_0/c_n$,
so the ratio we are seeking is $c_{n-1}/c_0$.
By inspection,
\begin{align*}
c_{n-1} &= n i^{n-1} - \omega n (-i)^{n-1} = n i^{n-1} (1-\omega)\\
c_0 &= 1 - \omega
\end{align*}
so
\[
\frac{a_1+ \cdots + a_n}{a_1 \cdots a_n}
= \begin{cases} n & n \equiv 1 \pmod{4} \\ -n & n \equiv 3 \pmod{4}.
\end{cases}
\]

\textbf{Remark:} The same argument shows that the ratio between
any two \emph{odd} elementary
symmetric functions of $a_1, \dots, a_n$ is independent
of $\theta$.

\item[A--6]
\textbf{First solution:}
(by Daniel Kane)
The probability is $1 - \frac{35}{12\pi^2}$.
We start with some notation and simplifications.
For simplicity, we
assume without loss of generality that the circle has radius 1.
Let $E$ denote the expected value of a random variable over all
choices of $P,Q,R$.
Write $[XYZ]$ for the area of triangle $XYZ$.

If $P,Q,R,S$ are the four points, we may ignore the case where three
of them are collinear, as this occurs with probability zero. Then the only
way they can fail to form the vertices of a convex quadrilateral is if one
of them lies inside the triangle formed by the other three. There are four
such configurations, depending on which point lies inside the triangle, and
they are mutually exclusive. Hence the desired probability is 1 minus
four times the probability that $S$ lies inside triangle $PQR$. That latter
probability is simply $E([PQR])$ divided by the area of
the disc.

Let $O$ denote the center of the circle,
and let $P',Q',R'$ be the projections of $P,Q,R$ onto the circle from $O$.
We can write
\[
[PQR] = \pm [OPQ] \pm [OQR] \pm [ORP]
\]
for a suitable choice of signs, determined as follows. If the points
$P',Q',R'$ lie on no semicircle, then all of the signs are positive.
If $P',Q',R'$ lie on a semicircle in that order and
$Q$ lies inside the triangle $OPR$, then the sign on $[OPR]$ is
positive and the others are negative.
If $P',Q',R'$ lie on a semicircle in that order and
$Q$ lies outside the triangle $OPR$, then the sign on $[OPR]$ is
negative and the others are positive.

We first calculate
\[
E([OPQ] + [OQR] + [ORP]) = 3 E([OPQ]).
\]
Write $r_1 = OP, r_2 = OQ, \theta = \angle POQ$, so that
\[
[OPQ] = \frac{1}{2} r_1 r_2 (\sin \theta).
\]
The distribution of $r_1$ is given by $2r_1$ on $[0,1]$
(e.g., by the change of variable formula to polar coordinates),
and similarly for $r_2$.
The distribution of $\theta$ is uniform on $[0,\pi]$.
These three distributions are independent; hence
\begin{align*}
& E([OPQ]) \\
&= \frac{1}{2} \left( \int_0^{1} 2r^2\,dr \right)^2
\left( \frac{1}{\pi} \int_0^\pi \sin (\theta)\,d\theta \right) \\
&= \frac{4}{9 \pi},
\end{align*}
and
\[
E([OPQ] + [OQR] + [ORP]) = \frac{4}{3 \pi}.
\]

We now treat the case where $P',Q',R'$ lie on a semicircle in
that order.
Put $\theta_1 = \angle POQ$ and $\theta_2 = \angle QOR$; then
the distribution of $\theta_1, \theta_2$ is uniform on the region
\[
0 \leq \theta_1, \quad 0 \leq \theta_2, \quad \theta_1 + \theta_2 \leq \pi.
\]
In particular, the distribution on $\theta = \theta_1 + \theta_2$
is $\frac{2\theta}{\pi^2}$ on $[0, \pi]$.
Put $r_P = OP, r_Q = OQ, r_R = OR$. Again, the distribution on $r_P$
is given by $2 r_P$ on $[0,1]$, and similarly for $r_Q, r_R$; these
are independent from each other and from the joint distribution
of $\theta_1,\theta_2$.
Write $E'(X)$ for the expectation of a random variable $X$
restricted to this part of the domain.

Let $\chi$ be the random variable with value 1 if $Q$ is inside
triangle $OPR$ and 0 otherwise.
We now compute
\begin{align*}
&E'([OPR]) \\
&=
\frac{1}{2} \left( \int_0^1 2r^2\,dr \right)^2
\left( \int_0^\pi \frac{2\theta}{\pi^2} \sin(\theta) \,d\theta \right)\\
&= \frac{4}{9 \pi} \\
& E'(\chi [OPR]) \\
&= E'(2 [OPR]^2 / \theta) \\
&=
\frac{1}{2} \left( \int_0^1 2r^3\,dr \right)^2
\left( \int_0^\pi \frac{2\theta}{\pi^2} \theta^{-1} \sin^2(\theta) \,d\theta \right)\\
&= \frac{1}{8\pi}.
\end{align*}
Also recall that given any triangle $XYZ$, if $T$ is chosen uniformly
at random inside $XYZ$, the expectation of $[TXY]$ is the area of
triangle bounded by $XY$ and the centroid of $XYZ$, namely
$\frac{1}{3} [XYZ]$.

Let $\chi$ be the random variable with value 1 if $Q$ is inside
triangle $OPR$ and 0 otherwise. Then
\begin{align*}
&E'([OPQ] + [OQR] + [ORP] - [PQR]) \\
&= 2 E'(\chi ([OPQ] + [OQR]) + 2 E'((1-\chi)[OPR]) \\
&= 2 E'(\frac{2}{3} \chi [OPR]) + 2 E'([OPR]) - 2 E'(\chi [OPR]) \\
&= 2E'([OPR]) - \frac{2}{3} E'(\chi [OPR]) = \frac{29}{36 \pi}.
\end{align*}
Finally, note that the case when $P',Q',R'$
lie on a semicircle in some order occurs with probability $3/4$.
(The case where they lie on a semicircle proceeding clockwise from $P'$
to its antipode has probability 1/4; this case and its two analogues are
exclusive and exhaustive.) Hence
\begin{align*}
&E([PQR]) \\
&= E([OPQ]+[OQR]+[ORP]) \\
&\quad - \frac{3}{4} E'([OPQ] + [OQR] + [ORP] - [PQR]) \\
&= \frac{4}{3 \pi} - \frac{29}{48 \pi} = \frac{35}{48 \pi},
\end{align*}
so the original probability is
\[
1 - \frac{4 E([PQR])}{\pi} = 1 - \frac{35}{12 \pi^2}.
\]

\textbf{Second solution:}
(by David Savitt)
As in the first solution, it suffices to check that for
$P,Q,R$ chosen uniformly at random in the disc, $E([PQR]) = \frac{35}{48 \pi}$.
Draw the lines $PQ, QR, RP$, which with probability 1 divide the interior
of the circle into seven regions. Put $a = [PQR]$, let $b_1,b_2,b_3$
denote the areas of the
three other regions sharing a side with the triangle, and let
$c_1,c_2,c_3$ denote the areas of the other three regions.
Put $A = E(a)$, $B = E(b_1)$, $C = E(c_1)$, so that
$A + 3B + 3C = \pi$.

Note that $c_1 + c_2 + c_3 + a$ is the area of the region in which we can
choose a fourth point $S$ so that the quadrilateral $PQRS$ fails to be
convex. By comparing expectations, we have $3C + A = 4A$,
so $A = C$ and $4A + 3B = \pi$.

We will compute $B + 2A = B + 2C$, which is the expected area of the part
of the circle cut off by a chord through two random points $D,E$, on the
side of the chord not containing a third random point $F$.
Let $h$ be the distance from the center $O$ of the circle to the line $DE$.
We now determine the distribution of $h$.

Put $r = OD$; the distribution of $r$ is $2r$ on $[0,1]$.
Without loss of generality, suppose $O$ is the origin and
$D$ lies on the positive $x$-axis.
For fixed $r$, the distribution of $h$ runs over $[0,r]$,
and can be computed as the area of the infinitesimal region in which
$E$ can be chosen so the chord through $DE$ has distance to $O$
between $h$ and $h+dh$, divided by $\pi$.
This region splits into two symmetric pieces, one of which lies
between chords making angles of $\arcsin(h/r)$ and
$\arcsin((h + dh)/r)$ with the $x$-axis.
The angle between these is $d\theta = dh/(r^2 - h^2)$.
Draw the chord through $D$ at distance $h$ to $O$, and let $L_1,L_2$ be the
lengths of the parts on opposite sides of $D$; then
the area we are looking for is $\frac{1}{2}(L_1^2 + L_2^2) d\theta$.
Since
\[
\{L_1, L_2 \} = \sqrt{1-h^2} \pm \sqrt{r^2 - h^2},
\]
the area we are seeking (after doubling) is
\[
2\frac{1 + r^2 - 2h^2}{\sqrt{r^2 - h^2}}.
\]
Dividing by $\pi$, then integrating over $r$, we compute the distribution
of $h$ to be
\begin{align*}
&\frac{1}{\pi} \int_h^1 2 \frac{1 + r^2 - 2h^2}{\sqrt{r^2 - h^2}} 2r\,dr \\
&= \frac{16}{3\pi} (1-h^2)^{3/2}.
\end{align*}

We now return to computing $B +2A$.
Let $A(h)$ denote the smaller of the two areas of the disc cut off
by a chord at distance $h$.
The chance that the third point is in the smaller (resp.\
larger) portion is $A(h)/\pi$ (resp.\ $1 - A(h)/\pi$),
and then the area we are trying to compute is $\pi - A(h)$
(resp.\ $A(h)$).
Using the distribution on $h$,
and the fact that
\begin{align*}
A(h) &= 2 \int_h^1 \sqrt{1-h^2}\,dh \\
&= \frac{\pi}{2}  - \arcsin(h) - h \sqrt{1-h^2},
\end{align*}
we find
\begin{align*}
&B+2A \\
&= \frac{2}{\pi} \int_0^1 A(h) (\pi - A(h))\, \frac{16}{3\pi} (1-h^2)^{3/2}
\,dh \\
&= \frac{35 + 24 \pi^2}{72 \pi}.
\end{align*}
Since $4A + 3B = \pi$, we solve to obtain
$A = \frac{35}{48 \pi}$ as in the first solution.

\textbf{Third solution:} (by Noam Elkies)
Again, we reduce to computing the average area of a triangle formed by
three random points $A,B,C$ inside a unit circle.
Let $O$ be the center of the circle, and put $c = \max\{OA,OB,OC\}$;
then the probability that $c \leq r$ is $(r^2)^3$, so the distribution
of $c$ is $6c^5\,dc$ on $[0,1]$.

Given $c$, the expectation of $[ABC]$ is equal to $c^2$ times $X$, the expected
area of a triangle formed by two random points $P,Q$ in a circle and
a fixed point $R$ on the boundary. We introduce polar coordinates centered
at $R$, in which the circle is given by $r = 2 \sin \theta$ for
$\theta \in [0, \pi]$. The distribution of a random point in that circle is
$\frac{1}{\pi} r\,dr\,d\theta$ over $\theta \in [0,\pi]$ and
$r \in [0, 2 \sin \theta]$. If $(r,\theta)$ and $(r',\theta')$ are the
two random points, then the area is $\frac{1}{2} rr' \sin |\theta - \theta'|$.

Performing the integrals over $r$ and $r'$ first, we find
\begin{align*}
X &= \frac{32}{9 \pi^2} \int_0^\pi \int_0^\pi \sin^3 \theta \sin^3 \theta'
\sin |\theta-\theta'|\,d\theta'\,d\theta \\
&= \frac{64}{9 \pi^2} \int_0^\pi \int_0^\theta \sin^3 \theta \sin^3 \theta'
\sin (\theta-\theta') \,d\theta'\,d\theta.
\end{align*}
This integral is unpleasant but straightforward; it yields
$X = 35/(36 \pi)$, and
$E([PQR]) = \int_0^1 6c^7 X\,dc = 35/(48 \pi)$, giving the desired
result.

\textbf{Remark:}
This is one of the oldest problems in geometric probability; it is an instance
of Sylvester's four-point problem, which nowadays is usually solved using
a device known as Crofton's formula.
We defer to \texttt{http://mathworld.wolfram.com/} for
further discussion.

\item[B--1]
The ``curve'' $x^3 + 3xy + y^3 - 1 = 0$ is actually reducible, because the
left side factors as
\[
(x+y-1)(x^2 -xy + y^2 + x + y + 1).
\]
Moreover, the second factor is
\[
\frac{1}{2} ((x+1)^2 + (y+1)^2 + (x-y)^2),
\]
so it only vanishes at $(-1,-1)$. Thus the curve in question consists of the
single point $(-1,-1)$ together with the line $x+y=1$. To form a triangle
with three points on this curve, one of its vertices must be $(-1,-1)$.
The other two vertices lie on the line $x+y=1$, so the length of the altitude
from $(-1,-1)$ is the distance from $(-1,-1)$ to $(1/2,1/2)$, or
$3 \sqrt{2}/2$. The area of an equilateral triangle of height $h$ is
$h^2 \sqrt{3}/3$, so the desired area is
$3 \sqrt{3}/2$.

\textbf{Remark:} The factorization used above is a special case of the fact
that
\begin{align*}
&x^3 + y^3 + z^3 - 3xyz \\
&= (x+y+z)(x+\omega y + \omega^2 z)(x + \omega^2 y
+ \omega z),
\end{align*}
where $\omega$ denotes a primitive cube root of unity. That fact in turn follows
from the evaluation of the determinant of the \emph{circulant matrix}
\[
\begin{pmatrix} x & y & z \\ z & x & y \\ y & z & x
\end{pmatrix}
\]
by reading off the eigenvalues of the eigenvectors
$(1, \omega^i, \omega^{2i})$ for $i=0,1,2$.

\item[B--2]
Let $\{x\} = x - \lfloor x \rfloor$ denote the fractional part of $x$.
For $i=0,\dots, n$, put $s_i = x_1 + \cdots + x_i$ (so that $s_0 = 0$).
Sort the numbers $\{s_0\}, \dots, \{s_n\}$ into ascending order,
and call the result $t_0, \dots, t_n$. Since $0 = t_0 \leq \cdots \leq
t_n < 1$, the differences
\[
t_1 - t_0, \dots, t_n - t_{n-1}, 1 - t_n
\]
are nonnegative and add up to 1. Hence (as in  the pigeonhole principle) one
of these differences
is no more than $1/(n+1)$; if it is anything other than $1 - t_n$,
it equals $\pm (\{s_i\} - \{s_j\})$ for some
$0 \leq i < j \leq n$. Put $S = \{x_{i+1}, \dots, x_j\}$ and
$m = \lfloor s_i \rfloor - \lfloor s_j \rfloor$; then
\begin{align*}
\left| m + \sum_{s \in S} s \right|
&= |m + s_j - s_i| \\
&= |\{s_j\} - \{s_i\}| \\
&\leq \frac{1}{n+1},
\end{align*}
as desired. In case $1 - t_n \leq 1 / (n+1)$, we take
$S = \{x_1, \dots, x_n\}$ and $m = -\lceil s_n \rceil$, and again obtain
the desired conclusion.

\item[B--3]
The maximum is $\binom{n}{2} + 1$, achieved for instance by a
convex $n$-gon: besides the trivial partition (in which all of the points
are in one part), each linear
partition occurs by drawing a line crossing a unique pair
of edges.

\textbf{First solution:}
We will prove that $L_S = \binom{n}{2} + 1$ in any configuration in which
no two of the lines joining points of $S$ are parallel. This suffices
to imply the maximum in all configurations: given a maximal configuration,
we may vary the points slightly to get another maximal configuration in which
our hypothesis is satisfied.
For convenience, we assume $n \geq 3$, as the cases $n=1,2$ are easy.

Let $P$ be the line at infinity in the real projective plane; i.e., $P$
is the set of possible directions of lines in the plane, viewed as a circle.
Remove the directions corresponding to lines through two points of $S$;
this leaves behind $\binom{n}{2}$ intervals.

Given a direction in one of the intervals, consider the set of linear
partitions achieved by lines parallel to that direction. Note that the
resulting collection of partitions depends only on the interval. Then
note that the collections associated to adjacent intervals differ in only
one element.

The trivial partition that puts all of $S$ on one side is in every such
collection. We now observe that for any other linear partition
$\{A,B\}$, the set of intervals to which $\{A,B\}$ is:
\begin{enumerate}
\item[(a)] a consecutive block of intervals, but
\item[(b)] not all of them.
\end{enumerate}
For (a), note that if $\ell_1, \ell_2$ are nonparallel lines achieving
the same partition, then we can rotate around their point of intersection
to achieve all of the intermediate directions on one side or the other.
For (b), the case $n=3$ is evident; to reduce the general case to this case,
take points $P,Q,R$ such that $P$ lies on the opposite side of
the partition from $Q$ and $R$.

It follows now that that each linear partition,
except for the trivial one, occurs in exactly one place as the partition
associated to some interval but not to its immediate counterclockwise neighbor.
In other words, the number of linear partitions is one more  than the
number of intervals, or $\binom{n}{2} + 1$ as desired.

\textbf{Second solution:}
We prove the upper bound
by induction on $n$. Choose a point $P$ in the convex hull of $S$.
Put $S' = S \setminus \{P\}$;
by the induction hypothesis, there are at most $\binom{n-1}{2} + 1$
linear partitions of $S'$. Note that each linear partition of $S$ restricts
to a linear partition of $S'$. Moreover, if two linear partitions of $S$
restrict to the same linear partition of $S'$, then that partition of $S'$
is achieved by a line through $P$.

By rotating a line through $P$, we see that there are at most $n-1$
partitions of $S'$ achieved by lines through $P$: namely, the partition only
changes when the rotating line passes through one of the points of $S$.
This yields the desired result.

\textbf{Third solution:} (by Noam Elkies) We enlarge the plane to a projective
plane by adding a line at infinity, then apply the polar duality map
centered at one of the points $O \in S$. This turns the rest of $S$ into
a set $S'$ of $n-1$ lines in the dual projective plane. Let $O'$ be the
point in the dual plane corresponding to the original line at infinity;
it does not lie on any of the lines in $S'$.

Let $\ell$ be a line in the original plane, corresponding to a point $P$ in
the dual plane. If we form the linear partition induced by $\ell$, then
the points of $S \setminus \{O\}$ lying in the same part as $O$
correspond to the lines of $S'$ which cross the segment $O'P$.
If we consider the dual affine plane as being divided into regions by
the lines of $S'$, then the lines of $S'$ crossing the segment $O'P$
are determined by which region $P$ lies in.

Thus our original maximum is equal to the maximum number of regions into
which $n-1$ lines divide an affine plane. By induction on $n$, this number
is easily seen to be $1 + \binom{n}{2}$.

\textbf{Fourth solution:} (by Florian Herzig)
Say that an \emph{$S$-line} is a line that intersects $S$ in at least two points.
We claim that the nontrivial linear partitions of $S$ are in natural bijection with pairs
$(\ell, \{X,Y\})$ consisting of an $S$-line $\ell$ and a nontrivial linear partition $\{X,Y\}$ of $\ell \cap S$.
Since an $S$-line $\ell$ admits precisely $|\ell\cap S|-1 \le \binom{|\ell \cap S|}{2}$ nontrivial linear partitions,
the claim implies that $L_S \le \binom n2 + 1$ with equality iff no three points of $S$ are collinear.

Let $P$ be the line at infinity in the real projective plane. Given any nontrivial linear partition $\{A,B\}$ of $S$, the
set of lines inducing this partition is a proper, open, connected subset $I$ of $P$. (It is proper because it has to omit
directions of $S$-lines that pass through both parts of the partition and open because we can vary the separating line. It is
connected because if we have two such lines that aren't parallel, we can rotate through their point of intersection to
get all intermediate directions.) Among all $S$-lines that intersect both $A$ and $B$ choose a line $\ell$ whose direction is
minimal (in the clockwise direction) with respect to the interval $I$; also, pick an arbitrary line $\ell'$ that induces
$\{A,B\}$. By rotating $\ell'$ clockwise to $\ell$ about their point of intersection, we see that the direction
of $\ell$ is the least upper bound of $I$. (We can't hit any point of $S$ during the rotation because of the minimality
property of $\ell$.)  The line $\ell$ is in fact unique because if the (parallel) lines $pq$ and $rs$ are two choices for $\ell$,
with $p$, $q \in A$; $r$, $s \in B$, then one of the diagonals $ps$, $qr$ would contradict the minimality property of
$\ell$. To define the above bijection we send $\{A,B\}$ to $(\ell, \{A \cap \ell, B \cap \ell\})$.

Conversely, suppose that we are given an $S$-line $\ell$ and a nontrivial linear partition $\{X,Y\}$ of $\ell \cap S$.
Pick any point $p \in \ell$ that induces the partition $\{X,Y\}$. If we rotate the line $\ell$ about $p$ in the counterclockwise
direction by a sufficiently small amount, we get a nontrivial linear partitition of $S$ that is independent of all choices.
(It is obtained from the partition of $S-\ell$ induced by $\ell$ by adjoining $X$ to one part and $Y$ to the other.) This
defines a map in the other direction.

By construction these two maps are inverse to each other, and this proves the claim.


\textbf{Remark:}
Given a finite set $S$ of points in $\mathbb{R}^n$, a \emph{non-Radon partition}
of $S$ is a pair $(A,B)$
of complementary subsets that can be separated by
a hyperplane. \emph{Radon's theorem} states that if $\#S\geq n+2$, then not
every $(A,B)$ is a non-Radon partition. The result of this problem
has been greatly
extended, especially within the context of matroid theory and oriented
matroid theory. Richard Stanley suggests the following references:
T. H. Brylawski, A combinatorial
perspective on the Radon convexity theorem, \emph{Geom. Ded.} \textbf{5}
(1976),
459-466; and T. Zaslavsky, Extremal arrangements of hyperplanes,
\emph{Ann. N. Y. Acad. Sci.} \textbf{440} (1985), 69-87.

\item[B--4]
The maximum is $2^k$, achieved for instance by the subspace
\[
\{(x_1, \dots, x_n) \in \mathbb{R}^n: x_1 = \cdots = x_{n-k} = 0\}.
\]

\textbf{First solution:}
More generally, we show that any affine $k$-dimensional plane in
$\mathbb{R}^n$ can contain at most $2^k$ points in $Z$. The proof is by
induction on $k+n$; the case $k=n=0$ is clearly true.

Suppose that $V$ is a $k$-plane in $\mathbb{R}^n$. Denote the
hyperplanes $\{x_n = 0\}$ and $\{x_n = 1\}$ by $V_0$ and $V_1$,
respectively. If $V\cap V_0$ and $V\cap V_1$ are each at most
$(k-1)$-dimensional, then $V\cap V_0\cap Z$ and $V\cap V_1 \cap Z$ each
have cardinality at most $2^{k-1}$ by the induction assumption, and
hence $V\cap Z$ has at most $2^k$ elements. Otherwise, if $V\cap V_0$ or
$V\cap V_1$ is $k$-dimensional, then $V \subset V_0$ or $V\subset V_1$;
now apply the induction hypothesis on $V$, viewed as a subset of
$\mathbb{R}^{n-1}$ by dropping the last coordinate.

\textbf{Second solution:}
Let $S$ be a subset of $Z$ contained in a $k$-dimensional subspace of $V$.
This is equivalent to asking that any $t_1, \dots, t_{k+1} \in S$
satisfy a nontrivial linear dependence $c_1 t_1 + \cdots + c_{k+1} t_{k+1} = 0$
with $c_1, \dots, c_{k+1} \in \mathbb{R}$. Since $t_1, \dots, t_{k+1} \in
\mathbb{Q}^n$, given such a dependence we can always find another one with
$c_1, \dots, c_{k+1} \in \mathbb{Q}$; then by clearing denominators, we
can find one with $c_1, \dots, c_{k+1} \in \mathbb{Z}$ and not all having a
common factor.

Let $\mathbb{F}_2$ denote the field of two elements, and let
$\overline{S} \subseteq \mathbb{F}_2^n$ be the reductions modulo 2 of the points of
$S$. Then any $t_1, \dots, t_{k+1} \in \overline{S}$ satisfy a nontrivial
linear dependence, because we can take the dependence from the end of
the previous paragraph and reduce modulo 2. Hence $\overline{S}$ is contained
in a $k$-dimensional subspace of $\mathbb{F}_{2^n}$, and the latter has cardinality
exactly $2^k$. Thus $\overline{S}$ has at most $2^k$ elements, as does
$S$.

Variant (suggested by David Savitt): if $\overline{S}$ contained $k+1$
linearly independent elements, the $(k+1) \times n$ matrix formed by these
would have a nonvanishing maximal minor. The lift of that minor back to $\RR$
would also not vanish, so $S$ would contain $k+1$ linearly independent
elements.

\textbf{Third solution:} (by Catalin Zara)
Let $V$ be a $k$-dimensional subspace. Form the matrix whose rows are the elements
of $V \cap Z$; by construction, it has row rank at most $k$. It thus also has
column rank at most $k$; in particular, we can choose $k$ coordinates such that
each point of $V \cap Z$ is determined by those $k$ of its coordinates. Since
each coordinate of a point in $Z$ can only take two values, $V \cap Z$ can have
at most $2^k$ elements.

\textbf{Remark:} The proposers probably did not realize that this problem appeared
online about three months before the exam, at
\texttt{http://www.artofproblemsolving.com/ Forum/viewtopic.php?t=105991}. (It
may very well have also appeared even earlier.)

\item[B--5]
The answer is $1/16$. We have
\begin{align*}
&\int_0^1 x^2 f (x)\,dx - \int_0^1 x f(x)^2\,dx \\
&= \int_0^1 (x^3/4 - x
( f(x)-x/2)^2)\,dx \\
&\leq \int_0^1 x^3/4\,dx = 1/16,
\end{align*}
with equality when $f(x) = x/2$.

\item[B--6]
\textbf{First solution:}
We start with some easy
upper and lower bounds on $a_n$.
We write $O(f(n))$ and $\Omega(f(n))$ for functions $g(n)$ such that
$f(n)/g(n)$ and $g(n)/f(n)$, respectively, are bounded above.
Since $a_n$ is a nondecreasing sequence, $a_{n+1}-a_n$ is bounded above,
so $a_n = O(n)$. That means $a_n^{-1/k} = \Omega(n^{-1/k})$, so
\[
a_n = \Omega \left( \sum_{i=1}^n i^{-1/k} \right)
= \Omega(n^{(k-1)/k}).
\]
In fact, all we will need is that $a_n \to \infty$ as $n \to \infty$.

By Taylor's theorem with remainder, for $1 < m < 2$ and $x>0$,
\[
|(1+x)^m - 1 - mx| \leq \frac{m(m-1)}{2}x^2.
\]
Taking $m = (k+1)/k$ and $x = a_{n+1}/a_n = 1 + a_n^{-(k+1)/k}$, we obtain
\[
\left| a_{n+1}^{(k+1)/k} - a_n^{(k+1)/k} - \frac{k+1}{k} \right|
\leq \frac{k+1}{2k^2} a_n^{-(k+1)/k}.
\]
In particular,
\[
\lim_{n \to \infty} a_{n+1}^{(k+1)/k} - a_n^{(k+1)/k} = \frac{k+1}{k}.
\]

In general, if $x_n$ is a sequence with $\lim_{n \to \infty} x_n = c$, then
also
\[
\lim_{n \to \infty} \frac{1}{n} \sum_{i=1}^n x_i = c
\]
by Cesaro's lemma. Explicitly, for any $\epsilon > 0$, we can find $N$ such that
$|x_n - c| \leq \epsilon/2$ for $n \geq N$, and then
\[
\left| c - \frac{1}{n} \sum_{i=1}^n x_i \right|
\leq \frac{n-N}{n} \frac{\epsilon}{2} + \frac{N}{n} \left| \sum_{i=1}^N (c-x_i) \right|;
\]
for $n$ large, the right side is smaller than $\epsilon$.

In our case, we deduce that
\[
\lim_{n \to \infty} \frac{a_n^{(k+1)/k}}{n} = \frac{k+1}{k}
\]
and so
\[
\lim_{n \to \infty} \frac{a_n^{k+1}}{n^k} = \left(\frac{k+1}{k} \right)^k,
\]
as desired.

\textbf{Remark:}
The use of Cesaro's lemma above is the special case $b_n = n$
of the \emph{Cesaro-Stolz
theorem}: if $a_n,b_n$ are sequences such that $b_n$ is positive,
strictly increasing, and unbounded, and
\[
\lim_{n \to \infty} \frac{a_{n+1} - a_n}{b_{n+1} - b_n} = L,
\]
then
\[
\lim_{n \to \infty} \frac{a_n}{b_n} = L.
\]

\textbf{Second solution:}
In this solution, rather than applying Taylor's theorem with remainder
to $(1+x)^m$ for $1 < m < 2$ and $x > 0$, we only apply convexity to deduce
that $(1+x)^m \geq 1 + mx$. This gives
\[
a_{n+1}^{(k+1)/k} - a_n^{(k+1)/k} \geq \frac{k+1}{k},
\]
and so
\[
a_n^{(k+1)/k} \geq \frac{k+1}{k} n + c
\]
for some $c \in \RR$. In particular,
\[
\liminf_{n \to \infty} \frac{a_n^{(k+1)/k}}{n} \geq \frac{k+1}{k}
\]
and so
\[
\liminf_{n \to \infty} \frac{a_n}{n^{k/(k+1)}} \geq \left(\frac{k+1}{k} \right)^{k/(k+1)}.
\]
But turning this around, the fact that
\begin{align*}
&a_{n+1} - a_n \\
&= a_n^{-1/k} \\
&\leq \left(\frac{k+1}{k} \right)^{-1/(k+1)} n^{-1/(k+1)}
(1 + o(1)),
\end{align*}
where $o(1)$ denotes a function tending to 0 as $n \to \infty$,
yields
\begin{align*}
&a_n \\
&\leq
\left(\frac{k+1}{k} \right)^{-1/(k+1)} \sum_{i=1}^n i^{-1/(k+1)} (1 + o(1)) \\
&= \frac{k+1}{k} \left(\frac{k+1}{k} \right)^{-1/(k+1)} n^{k/(k+1)}(1 + o(1)) \\
&= \left( \frac{k+1}{k} \right)^{k/(k+1)} n^{k/(k+1)}(1 + o(1)),
\end{align*}
so
\[
\limsup_{n \to \infty} \frac{a_n}{n^{k/(k+1)}} \leq \left( \frac{k+1}{k}
\right)^{k/(k+1)}
\]
and this completes the proof.

\textbf{Third solution:}
We argue that $a_n \to \infty$ as in the first solution.
Write $b_n = a_n - L n^{k/(k+1)}$, for a value of $L$ to be determined
later.
We have
\begin{align*}
&b_{n+1} \\
 &= b_n + a_n^{-1/k} - L ((n+1)^{k/(k+1)} - n^{k/(k+1)}) \\
&= e_1 + e_2,
\end{align*}
where
\begin{align*}
e_1 &= b_n + a_n^{-1/k} - L^{-1/k} n^{-1/(k+1)} \\
e_2 &= L ((n+1)^{k/(k+1)} - n^{k/(k+1)}) \\
&\quad - L^{-1/k} n^{-1/(k+1)}.
\end{align*}
We first estimate $e_1$.
For $-1 < m < 0$, by the convexity of $(1+x)^m$
and $(1+x)^{1-m}$, we have
\begin{align*}
1 + mx &\leq (1+x)^m \\
&\leq 1 + mx (1+x)^{m-1}.
\end{align*}
Hence
\begin{align*}
-\frac{1}{k} L^{-(k+1)/k} n^{-1} b_n &\leq e_1 - b_n \\
&\leq
-\frac{1}{k} b_n a_n^{-(k+1)/k}.
\end{align*}
Note that both bounds have sign opposite to $b_n$; moreover,
by the bound $a_n = \Omega(n^{(k-1)/k})$, both bounds have absolutely
value strictly less than that of $b_n$ for $n$ sufficiently large. Consequently,
for $n$ large,
\[
|e_1| \leq |b_n|.
\]
We now work on $e_2$.
By Taylor's theorem
with remainder applied to $(1+x)^m$ for $x > 0$ and $0 < m < 1$,
\begin{align*}
1+mx &\geq (1+x)^m \\
&\geq 1 + mx + \frac{m(m-1)}{2} x^2.
\end{align*}
The ``main term'' of $L ((n+1)^{k/(k+1)} - n^{k/(k+1)})$
is $L \frac{k}{k+1} n^{-1/(k+1)}$. To make this coincide with
$L^{-1/k} n^{-1/(k+1)}$, we take
\[
L = \left( \frac{k+1}{k} \right)^{k/(k+1)}.
\]
We then find that
\[
|e_2| = O(n^{-2}),
\]
and because $b_{n+1} = e_1 + e_2$, we have
$|b_{n+1}| \leq |b_n| + |e_2|$. Hence
\[
|b_n| = O\left (\sum_{i=1}^n i^{-2} \right) = O(1),
\]
and so
\[
\lim_{n \to \infty} \frac{a_n^{k+1}}{n^k} = L^{k+1} = \left( \frac{k+1}{k} \right)^k.
\]

\textbf{Remark:}
The case $k=2$ appeared on the 2004 Romanian Olympiad (district level).

\textbf{Remark:}
One can make a similar argument for any sequence given by
$a_{n+1} = a_n + f(a_n)$, when $f$ is a \emph{decreasing} function.

\textbf{Remark:}
Richard Stanley suggests a heuristic for determining the asymptotic
behavior of sequences of this type: replace the given recursion
\[
a_{n+1} - a_n = a_n^{-1/k}
\]
by the differential equation
\[
y' = y^{-1/k}
\]
and determine the asymptotics of the latter.
\end{itemize}


\end{document}



