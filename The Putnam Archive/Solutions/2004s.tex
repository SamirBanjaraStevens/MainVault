\documentclass[amssymb,twocolumn,pra,10pt,aps]{revtex4-1}
\usepackage{mathptmx,amsmath}

\newcommand{\RR}{\mathbb{R}}
\newcommand{\CC}{\mathbb{C}}
\DeclareMathOperator{\lcm}{lcm}
\DeclareMathOperator{\sgn}{sgn}

\begin{document}
\title{Solutions to the 65th William Lowell Putnam Mathematical Competition \\
    Saturday, December 4, 2004}
\author{Kiran Kedlaya and Lenny Ng}
\noaffiliation
\maketitle

\begin{itemize}

\item[A--1]
Yes. Suppose otherwise. Then there would be an $N$ such that $S(N) < .8N$
and $S(N+1) > .8(N+1)$; that is, O'Keal's free throw percentage is under $80\%$
at some point, and after one subsequent free throw (necessarily made),
her percentage is over $80\%$. If she makes $m$ of her first $N$ free
throws, then $m/N < 4/5$ and $(m+1)/(N+1) > 4/5$. This means that $5m <
4n < 5m+1$, which is impossible since then $4n$ is an integer between the
consecutive integers $5m$ and $5m+1$.

\textbf{Remark:}
This same argument works for any fraction of the form
$(n-1)/n$ for some integer $n>1$, but not for any other real number
between $0$ and $1$.

\item[A--2]
\textbf{First solution:} (partly due to Ravi Vakil)
Yes, it does follow.
For $i=1,2$,
let $P_i, Q_i, R_i$ be the vertices of $T_i$ opposide the sides of length
$a_i, b_i, c_i$, respectively.

We first check the case where $a_1 = a_2$ (or $b_1 = b_2$ or $c_1 = c_2$,
by the same argument after relabeling).
Imagine $T_2$ as being drawn with the base $Q_2R_2$
horizontal and the point $P_2$ above the line $Q_2R_2$. We may then
position $T_1$ so that $Q_1 = Q_2$, $R_1 = R_2$, and $P_1$ lies above the line
$Q_1R_1 = Q_2R_2$. Then $P_1$ also lies inside the region bounded by the
circles through $P_2$ centered at $Q_2$ and $R_2$. Since $\angle Q_2$
and $\angle R_2$ are acute, the part of this region above the line
$Q_2R_2$ lies within $T_2$. In particular, the
distance from $P_1$ to the line $Q_2R_2$ is less than or equal to the
distance from $P_2$ to the line $Q_2R_2$; hence $A_1 \leq A_2$.

To deduce the general case, put
\[
r = \max\{a_1/a_2, b_1/b_2, c_1/c_2\}.
\]
Let $T_3$ be the triangle with sides $ra_2, rb_2, rc_2$, which
has area $r^2 A_2$. Applying the special case to $T_1$ and $T_3$,
we deduce that $A_1 \leq r^2 A_2$; since $r \leq 1$ by hypothesis,
we have $A_1 \leq A_2$ as desired.

\textbf{Remark:}
Another geometric argument in the case $a_1 = a_2$
is that since angles $\angle Q_2$ and $\angle R_2$
are acute, the perpendicular to $Q_2R_2$ through $P_2$ separates $Q_2$ from
$R_2$. If $A_1 > A_2$, then $P_1$ lies above the parallel to $Q_2R_2$
through $P_2$; if then it lies on or to the left of the vertical line
through $P_2$, we have $c_1 > c_2$ because the inequality holds
for both horizontal and vertical components (possibly with equality for one,
but not both). Similarly, if $P_1$ lies to the right of the vertical,
then $b_1 > b_2$.

\textbf{Second solution:} (attribution unknown)
Retain notation as in the first paragraph of the first solution.
Since the angle measures in any triangle add up to $\pi$, some angle
of $T_1$ must have measure less than or equal to its counterpart in
$T_2$. Without loss of generality assume that $\angle P_1 \leq \angle
P_2$. Since the latter is acute (because $T_2$ is acute), we have
$\sin \angle P_1 \leq \sin \angle P_2$. By the Law of Sines,
\[
A_1 = \frac{1}{2} b_1 c_1 \sin \angle P_1 \leq \frac{1}{2}
b_2 c_2 \sin \angle P_2 = A_2.
\]

\textbf{Remark:} Many other solutions are possible; for instance,
one uses Heron's formula for the area of a triangle in terms of
its side lengths.

\item[A--3]
Define a sequence $v_n$ by $v_n = (n-1)(n-3)\cdots(4)(2)$ if $n$ is
odd and $v_n = (n-1)(n-3)\cdots(3)(1)$ if $n$ is even; it suffices to
prove that $u_n = v_n$ for all $n \geq 2$. Now $v_{n+3} v_n =
(n+2)(n)(n-1)!$ and $v_{n+2}v_{n+1} = (n+1)!$, and so $v_{n+3} v_n -
v_{n+2} v_{n+1} = n!$. Since we can check that $u_n = v_n$ for
$n=2,3,4$, and $u_n$ and $v_n$ satisfy the same recurrence, it
follows by induction that $u_n = v_n$ for all $n\geq 2$, as desired.

\item[A--4]
It suffices to verify that
\[
x_1 \cdots x_n = \frac{1}{2^n n!} \sum_{e_i \in \{-1,1\}}
(e_1\cdots e_n) (e_1 x_1 + \cdots + e_n x_n)^n.
\]
To check this, first note that the right side vanishes identically
for $x_1 = 0$, because each term cancels the corresponding term with $e_1$
flipped. Hence the right side, as a polynomial, is divisible by $x_1$;
similarly it is divisible by $x_2, \dots, x_n$. Thus the right side
is equal to $x_1\cdots x_n$ times a scalar. (Another way to see this:
the right side is clearly odd as a polynomial in each individual variable,
but the only degree $n$ monomial in $x_1, \dots, x_n$ with that property
is $x_1 \cdots x_n$.) Since each summand
contributes $\frac{1}{2^n} x_1 \cdots x_n$ to the sum, the scalar factor is
1 and we are done.

\textbf{Remark:} Several variants on the above construction
are possible; for instance,
\[
x_1 \cdots x_n = \frac{1}{n!}
\sum_{e_i \in \{0,1\}} (-1)^{n - e_1 - \cdots - e_n}
(e_1 x_1 + \cdots + e_n x_n)^n
\]
by the same argument as above.

\textbf{Remark:} These construction work over any field of characteristic
greater than $n$ (at least for $n>1$).
On the other hand, no construction is possible over
a field of characteristic $p \leq n$, since the coefficient of
$x_1\cdots x_n$ in the expansion of
$(e_1 x_1 + \cdots + e_n x_n)^n$ is zero for any $e_i$.

\textbf{Remark:} Richard Stanley asks whether one can use fewer than
$2^n$ terms, and what the smallest possible number is.

\item[A--5]
\textbf{First solution:}
First recall that any graph with $n$ vertices and $e$ edges has at
least $n-e$ connected components (add each edge one at a time, and note
that it reduces the number of components by at most 1). Now
imagine the squares of the checkerboard as a graph, whose vertices are
connected if the corresponding squares share a side and are
the same color. Let
$A$ be the number of edges in the graph, and let $B$ be the number of
4-cycles (formed by monochromatic $2 \times 2$ squares).
If we remove the bottom edge of each 4-cycle, the resulting graph has
the same number of connected components as the original one;
hence this number is at least
\[
mn - A+B.
\]
By the linearity of expectation, the expected number of connected
components is at least
\[
mn - E(A) + E(B).
\]
Moreover, we may compute $E(A)$ by summing over the individual
pairs of adjacent squares, and we may compute $E(B)$ by summing over
the individual $2 \times 2$ squares. Thus
\begin{align*}
E(A) &= \frac{1}{2}(m(n-1) + (m-1)n), \\
E(B) &= \frac{1}{8}(m-1)(n-1),
\end{align*}
and so the expected number of components is at least
\begin{align*}
& mn - \frac{1}{2}(m(n-1)+(m-1)n) + \frac{1}{8}(m-1)(n-1) \\
&= \frac{mn + 3m + 3n + 1}{8} >
\frac{mn}{8}.
\end{align*}

\textbf{Remark:}
A ``dual'' approach is to consider
the graph whose vertices are the corners of the squares of the checkerboard,
with two vertices joined if they are adjacent and the edge between then
does not separate two squares of the same color. In this approach,
the 4-cycles become isolated vertices, and the bound on components
is replaced by a call to Euler's formula relating the vertices, edges
and faces of a planar figure. (One must be careful, however, to correctly
handle faces which are not simply connected.)

\textbf{Second solution:} (by Noam Elkies)
Number the squares of the checkerboard $1, \dots, mn$ by numbering the
first row from left to right, then the second row, and so on. We prove
by induction on $i$ that if we just consider the figure formed by the
first $i$ squares, its expected number of monochromatic components is
at least $i/8$. For $i=1$, this is clear.

Suppose the $i$-th square does not abut the left edge or the top row
of the board.
Then we may divide into three cases.
\begin{itemize}
\item
With probability $1/4$, the $i$-th square is opposite in color from
the adjacent squares directly above and to the left of it. In this case
adding the $i$-th square adds one component.
\item
With probability $1/8$, the $i$-th square is the same in color as the
adjacent squares directly above and to the left of it, but opposite in
color from its diagonal neighbor above and to the left. In this case,
adding the $i$-th square either removes a component or leaves the number
unchanged.
\item
In all other cases, the number of components remains unchanged upon
adding the $i$-th square.
\end{itemize}
Hence adding the $i$-th square increases the expected number of components
by $1/4 - 1/8 = 1/8$.

If the $i$-th square does abut the left edge of the board, the situation
is even simpler: if the $i$-th square differs in color from the square
above it, one component is added, otherwise the number does not change. Hence
adding the $i$-th square increases the expected number of components
by $1/2$; likewise if the $i$-th square abuts the top edge of the board.
Thus the expected number of components is at least $i/8$ by induction,
as desired.

\textbf{Remark:} Some solvers attempted to consider adding one row
at a time, rather than one square; this must be handled with great
care, as it is possible that the number of components can drop
rather precipitously upon adding an entire row.

\item[A--6]
By approximating each integral with a Riemann sum, we may reduce to
proving the discrete analogue: for $x_{ij} \in \RR$ for
$i,j=1, \dots, n$,
\begin{multline*}
n \sum_{i=1}^n \left( \sum_{j=1}^n x_{ij} \right)^2
+ n \sum_{j=1}^n \left( \sum_{i=1}^n x_{ij} \right)^2 \\
\leq \left( \sum_{i=1}^n \sum_{j=1}^n x_{ij} \right)^2
+ n^2 \sum_{i=1}^n \sum_{j=1}^n x_{ij}^2.
\end{multline*}
The difference between the right side and the left side is
\[
\frac{1}{4} \sum_{i,j,k,l=1}^n (x_{ij} + x_{kl} - x_{il} - x_{kj})^2,
\]
which is evidently nonnegative. If you prefer not to discretize,
you may rewrite the original inequality as
\[
\int_0^1 \int_0^1 \int_0^1 \int_0^1 F(x,y,z,w)^2
\,dx\,dy\,dz\,dw \geq 0
\]
for
\[
F(x,y,z,w) = f(x,y) + f(z,w) - f(x,w) - f(z,y).
\]

\textbf{Remark:} (by Po-Ning Chen)
The discrete inequality can be arrived at more systematically
by repeatedly applying the following identity: for
any real $a_1, \dots, a_n$,
\[
\sum_{1 \leq i < j \leq n} (x_i - x_j)^2
= n \sum_{i=1}^n x_i^2 - \left( \sum_{i=1}^n x_i \right)^2.
\]

\textbf{Remark:} (by David Savitt)
The discrete inequality can also be interpreted as follows.
For $c,d \in \{1, \dots, n-1\}$ and $\zeta_n = e^{2\pi i/n}$, put
\[
z_{c,d} = \sum_{i,j} \zeta_n^{c i + d j} x_{ij}.
\]
Then the given inequality is equivalent to
\[
\sum_{c,d=1}^{n-1} |z_{c,d}|^2 \geq 0.
\]

\item[B--1]
Let $k$ be an integer, $0\leq k\leq n-1$. Since $P(r)/r^k = 0$, we
have
\begin{multline*}
c_n r^{n-k} + c_{n-1} r^{n-k+1} + \dots + c_{k+1} r \\
= - (c_k + c_{k-1} r^{-1} + \dots + c_0 r^{-k}).
\end{multline*}
Write $r = p/q$ where $p$ and $q$ are relatively prime. Then the
left hand side of the above equation can be written as a fraction
with denominator $q^{n-k}$, while the right hand side is a fraction
with denominator $p^k$. Since $p$ and $q$ are relatively prime, both
sides of the equation must be an integer, and the result follows.

\textbf{Remark:}
If we write $r = a/b$ in lowest terms, then $P(x)$ factors as $(bx-a)Q(x)$,
where the polynomial $Q$ has integer coefficients because you can
either do the long division from the left and get denominators divisible only
by primes dividing $b$, or do it from the right and get denominators
divisible only by primes dividing $a$. The numbers given in the problem
are none other than $a$ times the coefficients of $Q$.
More generally, if $P(x)$ is divisible,
as a polynomial over the rationals, by a polynomial $R(x)$ with integer
coefficients, then $P/R$ also has integer coefficients; this is known
as ``Gauss's lemma'' and holds in any unique factorization domain.

\item[B--2]
\textbf{First solution:}
We have
\[
(m+n)^{m+n} > \binom{m+n}{m} m^m n^n
\]
because the binomial expansion of $(m+n)^{m+n}$ includes the term on
the right as well as some others. Rearranging this inequality yields
the claim.

\textbf{Remark:}
One can also interpret this argument combinatorially.
Suppose that we choose $m+n$ times (with replacement) uniformly
randomly from a set of $m+n$ balls, of which $m$ are red and $n$ are
blue. Then the probability of picking each ball exactly once is
$(m+n)!/(m+n)^{m+n}$. On the other hand, if $p$ is the probability
of picking exactly $m$ red balls, then $p<1$ and the probability
of picking each ball exactly once is $p (m^m/m!)(n^n/n!)$.

\textbf{Second solution:} (by David Savitt)
Define
\[
S_k = \{i/k: i=1, \dots, k\}
\]
and rewrite the desired inequality as
\[
\prod_{x \in S_m} x \prod_{y \in S_n} y > \prod_{z \in S_{m+n}} z.
\]
To prove this, it suffices to check that if we sort the multiplicands
on both sides into increasing order, the $i$-th term on the left
side is greater than or equal to the $i$-th term on the right side.
(The equality is strict already for $i=1$, so you do get a strict inequality
above.)

Another way to say this is that for
any $i$, the number of factors on the left side which are less than
$i/(m+n)$ is less than $i$. But since $j/m < i/(m+n)$ is equivalent to
$j < im/(m+n)$, that number is
\begin{align*}
&\left\lceil \frac{im}{m+n} \right\rceil -1 +
\left\lceil \frac{in}{m+n} \right\rceil -1 \\
&\leq \frac{im}{m+n} + \frac{in}{m+n} - 1 = i-1.
\end{align*}

\textbf{Third solution:}
Put $f(x) = x (\log (x+1) - \log x)$; then for $x>0$,
\begin{align*}
f'(x) &= \log(1 + 1/x) - \frac{1}{x+1} \\
f''(x) &= - \frac{1}{x(x+1)^2}.
\end{align*}
Hence $f''(x) < 0$ for all $x$; since $f'(x) \to 0$ as $x \to \infty$,
we have $f'(x) > 0$ for $x>0$, so $f$ is strictly increasing.

Put $g(m) = m \log m - \log(m!)$; then $g(m+1) - g(m) = f(m)$,
so $g(m+1)-g(m)$ increases with $m$. By induction,
$g(m+n) - g(m)$ increases with $n$ for any positive integer $n$,
so in particular
\begin{align*}
g(m+n) - g(m) &> g(n) - g(1) + f(m) \\
&\geq g(n)
\end{align*}
since $g(1) = 0$. Exponentiating yields the desired inequality.

\textbf{Fourth solution:} (by W.G. Boskoff and Bogdan Suceav\u{a})
We prove the claim by induction on $m+n$. The base case is $m=n=1$, in which case
the desired inequality is obviously true: $2!/2^2 = 1/2 < 1 = (1!/1^1)(1!/1^1)$.
To prove the induction step, suppose $m+n > 2$; we must then have $m>1$ or $n>1$ or both.
Because the desired result is symmetric in $m$ and $n$, we may as well assume $n > 1$.
By the induction hypothesis, we have
\[
\frac{(m+n-1)!}{(m+n-1)^{m+n-1}} < \frac{m!}{m^m} \frac{(n-1)!}{(n-1)^{n-1}}.
\]
To obtain the desired inequality, it will suffice to check that
\[
\frac{(m+n-1)^{m+n-1}}{(m+n-1)!} \frac{(m+n)!}{(m+n)^{m+n}}< \frac{(n-1)^{n-1}}{(n-1)!} \frac{n!}{(n)^{n}}
\]
or in other words
\[
\left( 1 - \frac{1}{m+n} \right)^{m+n-1} < \left(1 - \frac{1}{n} \right)^{n-1}.
\]
To show this, we check that the function $f(x) = (1 - 1/x)^{x-1}$
is strictly decreasing for $x>1$; while this can be achieved using the weighted arithmetic-geometric mean
inequality, we give a simple calculus proof instead. The derivative of $\log f(x)$ is
$\log (1-1/x) + 1/x$, so it is enough to check that this is negative for $x>1$.
An equivalent statement is that $\log (1-x) + x < 0$ for $0 < x < 1$;
this in turn holds because the function $g(x) = \log(1-x) + x$ tends to 0 as $x \to 0^+$
and has derivative $1 - \frac{1}{1-x} < 0$ for $0 < x < 1$.

\item[B--3]
The answer is $\{a\,|\,a>2\}$. If $a>2$, then the function $f(x) =
2a/(a-2)$ has the desired property; both perimeter and area of $R$
in this case are $2a^2/(a-2)$. Now suppose that $a\leq 2$, and let
$f(x)$ be a nonnegative continuous function on $[0,a]$. Let
$P=(x_0,y_0)$ be a point on the graph of $f(x)$ with maximal
$y$-coordinate; then the area of $R$ is at most $ay_0$ since it lies
below the line $y=y_0$. On the other hand, the points $(0,0)$,
$(a,0)$, and $P$ divide the boundary of $R$ into three sections. The
length of the section between $(0,0)$ and $P$ is at least the
distance between $(0,0)$ and $P$, which is at least $y_0$; the
length of the section between $P$ and $(a,0)$ is similarly at least
$y_0$; and the length of the section between $(0,0)$ and $(a,0)$ is
$a$. Since $a\leq 2$, we have $2y_0 + a > ay_0$ and hence the
perimeter of $R$ is strictly greater than the area of $R$.

\item[B--4]
\textbf{First solution:}
Identify the $xy$-plane with the complex plane $\mathbb{C}$, so that
$P_k$ is the real number $k$. If $z$ is sent to $z'$ by a
counterclockwise rotation by $\theta$ about $P_k$, then $z'-k =
e^{i\theta} (z-k)$; hence the rotation $R_k$ sends $z$ to $\zeta z +
k (1-\zeta)$, where $\zeta = e^{2\pi i/n}$. It follows that $R_1$
followed by $R_2$ sends $z$ to $\zeta(\zeta z +(1-\zeta)) + 2
(1-\zeta) = \zeta^2 z + (1-\zeta)(\zeta + 2)$, and so forth; an easy
induction shows that $R$ sends $z$ to
\[
\zeta^n z + (1-\zeta)(\zeta^{n-1} + 2 \zeta^{n-2} + \dots + (n-1)
\zeta + n).
\]
Expanding the product $(1-\zeta)(\zeta^{n-1} + 2 \zeta^{n-2} + \dots
+ (n-1) \zeta + n)$ yields $-\zeta^n - \zeta^{n-1} - \dots - \zeta +
n = n$. Thus $R$ sends $z$ to $z+n$; in cartesian coordinates,
$R(x,y) = (x+n,y)$.

\textbf{Second solution:}
(by Andy Lutomirski, via Ravi Vakil)
Imagine a regular $n$-gon of side length 1 placed with its top edge
on the $x$-axis and the left endpoint of that edge
at the origin. Then the rotations
correspond to rolling this $n$-gon along the $x$-axis; after the
$n$ rotations, it clearly ends up in its original rotation and translated
$n$ units to the right. Hence the whole plane must do so as well.

\textbf{Third solution:} (attribution unknown)
Viewing each $R_k$ as a function of a complex number $z$ as in the
first solution, the function $R_n \circ R_{n-1} \circ \cdots \circ
R_1(z)$ is linear in $z$ with slope $\zeta^n = 1$. It thus equals
$z + T$ for some $T \in \CC$. Since $f_1(1) = 1$, we can write
$1+T = R_n \circ \cdots \circ R_2(1)$. However, we also have
\[
R_n \circ \cdots \circ R_2(1) = R_{n-1} \circ R_1(0) + 1
\]
by the symmetry in how the $R_i$ are defined. Hence
\[
R_n(1+T) = R_n \circ R_1(0) + R_n(1) = T + R_n(1);
\]
that is, $R_n(T) = T$. Hence $T = n$, as desired.

\item[B--5]
\textbf{First solution:}
By taking logarithms, we see that the desired limit is $\exp(L)$,
where $L = \lim_{x\to 1^-} \sum_{n=0}^{\infty} x^n \left(
\ln(1+x^{n+1}) - \ln(1+x^n) \right)$. Now
\begin{align*}
&\sum_{n=0}^N x^n \left( \ln(1+x^{n+1}) - \ln(1+x^n) \right) \\
& = 1/x
\sum_{n=0}^N x^{n+1} \ln(1+x^{n+1}) - \sum_{n=0}^N x^n\ln(1+x^n) \\
&=
x^N \ln(1+x^{N+1}) - \ln 2 + (1/x-1) \sum_{n=1}^N x^n\ln(1+x^n);
\end{align*}
since $\lim_{N\to\infty} (x^N\ln(1+x^{N+1})) = 0$ for $0<x<1$, we
conclude that $L = - \ln 2 + \lim_{x\to 1^-} f(x)$, where
\begin{align*}
f(x) &= (1/x-1) \sum_{n=1}^{\infty} x^n\ln(1+x^n) \\
&= (1/x-1)
\sum_{n=1}^\infty \sum_{m=1}^\infty (-1)^{m+1} x^{n+mn}/m.
\end{align*}
This final double sum converges absolutely when $0<x<1$, since
\begin{align*}
\sum_{n=1}^\infty \sum_{m=1}^\infty x^{n+mn}/m &= \sum_{n=1}^\infty
x^n (-\ln(1-x^n)) \\
&< \sum_{n=1}^\infty x^n (-\ln(1-x)),
\end{align*}
which converges. (Note that $-\ln(1-x)$ and $-\ln(1-x^n)$ are
positive.) Hence we may interchange the summations in $f(x)$ to
obtain
\begin{align*}
f(x) &= (1/x-1) \sum_{m=1}^\infty \sum_{n=1}^\infty \frac{(-1)^{m+1}
x^{(m+1)n}}{m} \\
&= (1/x-1) \sum_{m=1}^\infty \frac{(-1)^{m+1}}
{m}\left(\frac{x^m(1-x)}{1-x^{m+1}}\right).
\end{align*}
This last sum converges absolutely uniformly in $x$, so it is legitimate
to take limits term by term.
Since $\lim_{x\to 1^-} \frac{x^m{1-x}}{1-x^{m+1}} = \frac{1}{m+1}$
for fixed $m$, we have
\begin{align*}
\lim_{x\to 1^-} f(x) &= \sum_{m=1}^\infty \frac{(-1)^{m+1}}{m(m+1)} \\
&= \sum_{m=1}^\infty (-1)^{m+1}\left( \frac{1}{m}-\frac{1}{m+1} \right) \\
&= 2 \left( \sum_{m=1}^\infty \frac{(-1)^{m+1}}{m} \right) - 1 \\ &= 2
\ln 2 - 1,
\end{align*}
and hence $L = \ln 2 - 1$ and the desired limit is $2/e$.

\textbf{Remark:} Note that the last series is not absolutely convergent,
so the recombination must be done without rearranging terms.

\textbf{Second solution:}
(by Greg Price, via Tony Zhang and Anders Kaseorg)
Put $t_n(x) = \ln(1 + x^n)$;
we can then write $x^n = \exp(t_n(x)) - 1$, and
\[
L = \lim_{x \to 1^-} \sum_{n=0}^\infty (t_n(x) - t_{n+1}(x))(1 - \exp(t_n(x))).
\]
The expression on the right is a Riemann sum approximating the
integral $\int_0^{\ln 2} (1-e^t)\,dt$, over the subdivision of $[0,\ln(2))$
given by the $t_n(x)$. As $x \to 1^-$, the maximum difference between
consecutive $t_n(x)$ tends to 0, so the Riemann sum tends to the value
of the integral. Hence $L = \int_0^{\ln 2}(1 - e^t)\,dt = \ln 2 - 1$, as
desired.

\def\calA{\mathcal{A}}
\def\calB{\mathcal{B}}

\item[B--6]
\textbf{First solution:}
(based on a solution of Dan Bernstein)
Note that for any $b$, the condition that $b \notin \calB$ already
forces $\limsup N(x)/x$ to be at most 1/2: pair off $2mb + n$ with
$(2m+1)b+n$ for $n=1,\dots, b$, and note that at most one member of each
pair may belong to $\calA$.
The idea of the proof is to do something
similar with pairs replaced by larger clumps, using long runs of
excluded elements of $B$.

Suppose we have positive integers
$b_0 = 1, b_1, \dots, b_n$ with
the following properties:
\begin{enumerate}
\item[(a)] For $i=1, \dots, n$,
$c_i = b_i/(2b_{i-1})$ is an integer.
\item[(b)] For $e_i \in \{-1,0,1\}$,
$|e_1b_1 + \cdots + e_n b_n| \notin \calB$.
\end{enumerate}
Each nonnegative integer $a$ has a unique ``base expansion''
\[
a = a_0 b_0 + \cdots + a_{n-1} b_{n-1} + m b_n \qquad (0 \leq a_i < 2c_i);
\]
if two integers have expansions with the same value of $m$, and values
of $a_i$ differing by at most 1 for $i=0, \dots, n-1$, then their
difference is not in $\calB$, so at most one of them lies in $\calA$.
In particular,
for any $d_i \in \{0, \dots, c_i - 1\}$, any $m_0 \in \{0, 2c_0 - 1\}$
and any $m_n$, the set
\begin{multline*}
\{m_0 b_0 + (2d_1 +e_1)b_0 + \cdots \\
+ (2d_{n-1} + e_{n-1})b_{n-1} + (2m_n + e_n)b_n\},
\end{multline*}
where each $e_i$ runs over $\{0,1\}$,
contains at most one element of $\calA$; consequently,
$\limsup N(x)/x \leq 1/2^{n}$.

We now produce such $b_i$ recursively, starting with $b_0 = 1$ (and both
(a) and (b) holding vacuously).  Given $b_0, \dots, b_n$ satisfying (a)
and (b), note that $b_0 + \cdots + b_{n-1} < b_n$ by induction on $n$.
By the hypotheses of the problem, we can find a set $S_n$ of $6b_n$
consecutive integers, none of which belongs to $\calB$. Let $b_{n+1}$
be the second-smallest multiple of $2b_n$ in $S_n$; then $b_{n+1} + x \in
S_n$ for $-2b_n \leq x \leq 0$ clearly, and also for $0 \leq x \leq 2b_n$
because there are most $4b_n-1$ elements of $S_n$ preceding $b_{n+1}$.
In particular, the analogue of (b) with $n$ replaced by $n+1$ holds for
$e_{n+1} \neq 0$; of course it holds for $e_{n+1} = 0$ because (b) was
already known. Since the analogue of (a) holds by construction, we have
completed this step of the construction and the recursion may continue.

Since we can construct $b_0, \dots, b_n$ satisfying (a) and (b) for any $n$,
we have $\limsup N(x)/x \leq 1/2^n$ for any $n$, yielding $\lim N(x)/x =
0$ as desired.

\textbf{Second solution:} (by Paul Pollack)
Let $S$ be the set of possible values of $\limsup N(x)/x$;
since $S \subseteq [0,1]$ is bounded, it has a least upper bound $L$.
Suppose by way of contradiction that $L>0$;
we can then choose $\calA, \calB$ satisfying the conditions of the problem such
that $\limsup N(x)/x > 3L/4$.

To begin with, we can
certainly find some positive integer $m \notin \calB$, so that
$\calA$ is disjoint from $\calA + m = \{a +m: a \in \calA\}$.
Put $\calA' = \calA \cup (\calA + m)$ and let
$N'(x)$ be the size of $\calA' \cap \{1, \dots, x\}$; then
$\limsup N'(x)/x = 3L/2 > L$, so $\calA'$ cannot obey the conditions
of the problem statement. That is, if we let $\calB'$ be the set of positive
integers that occur as differences between elements of $\calA'$,
then there exists an integer $n$ such that among any $n$ consecutive
integers, at least one lies in $\calB'$.
But
\[
\calB' \subseteq \{b + em: b \in \calB, e \in \{-1,0,1\}\},
\]
so among any $n+2m$ consecutive integers, at least one lies in $\calB$.
This contradicts the condition of the problem statement.

We conclude that it is impossible to have $L>0$, so $L = 0$
and $\lim N(x)/x = 0$ as desired.

\textbf{Remark:} A hybrid between these two arguments is to
note that if we can produce $c_1, \dots, c_n$ such that
$|c_i - c_j| \notin \calB$ for $i,j = 1, \dots, n$, then
the translates $\calA + c_1, \dots, \calA + c_n$ are disjoint
and so $\limsup N(x)/x \leq 1/n$. Given $c_1 \leq \dots \leq c_n$
as above, we can then choose $c_{n+1}$ to be the largest element of a
run of $c_n+1$ consecutive integers, none of which lie in $\calB$.

\end{itemize}


\end{document}



