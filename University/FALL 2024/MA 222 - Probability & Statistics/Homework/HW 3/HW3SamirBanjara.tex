%%%%%%%%%%%%%%%%%%%%%%%%%%%%% Define Article %%%%%%%%%%%%%%%%%%%%%%%%%%%%%%%%%%
\documentclass{article}
%%%%%%%%%%%%%%%%%%%%%%%%%%%%%%%%%%%%%%%%%%%%%%%%%%%%%%%%%%%%%%%%%%%%%%%%%%%%%%%

%%%%%%%%%%%%%%%%%%%%%%%%%%%%% Using Packages %%%%%%%%%%%%%%%%%%%%%%%%%%%%%%%%%%
\usepackage{geometry}
\usepackage{graphicx}
\usepackage{amssymb}
\usepackage{amsmath}
\usepackage{amsthm}
\usepackage{empheq}
\usepackage{mdframed}
\usepackage{booktabs}
\usepackage{lipsum}
\usepackage{graphicx}
\usepackage{color}
\usepackage{psfrag}
\usepackage{pgfplots}
\usepackage{bm}


%%%%%%%%%%%%%%%%%%%%%%%%%%%%%%% Title & Author %%%%%%%%%%%%%%%%%%%%%%%%%%%%%%%%
\title{Homework \#3}
\author{Samir Banjara}
\date{\today}
%%%%%%%%%%%%%%%%%%%%%%%%%%%%%%%%%%%%%%%%%%%%%%%%%%%%%%%%%%%%%%%%%%%%%%%%%%%%%%%

% Custom environment for questions
\newenvironment{question}[1]{
    \vspace{1em}
    \noindent\textbf{#1} \\
    \vspace{0.5em}
}

\begin{document}
    \maketitle
\noindent
\(\textbf{Pledge:}\)
 I pledge my honor that I have abided by the Stevens Honor System.
\noindent
\textbf{Signature:} Samir Banjara

% Question 1
\section*{Problem 1}

\textbf{(A)} Find the probability of rolling two dice and not getting doubles.\\

\textbf{Solution:}
\vspace{1em} \\
Given the set of possibilities of rolling two dice,
\[
s = 
\begin{aligned}
&\left\{(1,1), (2,1), (3, 1), (4, 1), (5, 1), (6, 1)\right\}  \\
&\left\{(1,2), (2,2), (3, 2), (4, 2), (5, 2), (6, 2)\right\}  \\
&\left\{(1,3), (2, 3), (3, 3), (4, 3), (5, 3), (6, 3)\right\} \\
&\left\{(1, 4), (2, 4), (3, 4), (4, 4), (5, 4), (6, 4)\right\} \\
&\left\{(1, 5), (2, 5), (3, 5), (4, 5), (5, 5), (6, 5)\right\} \\
&\left\{(1, 6), (2, 6), (3, 6), (4, 6), (5, 6), (6, 6)\right\}
\end{aligned}
\]

There are 30 rolls out of 36 that are not an ordered double. Thus, the probability is \( \frac{30}{36} \).\\

\vspace{1em}
\textbf{(B)} Given that every fifth person in line will get a coupon for a free box of popcorn at the movies, what is the probability that you don’t get a coupon when you’re in line?\\

\textbf{Solution:}
\vspace{1em} \\
Every 5th person in line gets a coupon. That means 1 person out of 5 gets a coupon. The probability of not getting a coupon is then:

\[
1 - \frac{1}{5} = \frac{4}{5}
\]

\section*{Problem 2}

Insulin pens used to administer a patient’s insulin at hospitals have a malfunction rate of 9\%. This means that out of a box of 200 pens, 18 are defective. Find the probability of randomly selecting 3 defective insulin pens in a row from a brand-new box of 200 pens, if a defective pen is immediately discarded.

\textbf{Solution:} \\
The probability of selecting a defective pen out of a new box of 200 is \( \frac{18}{200} \). Since the pen is discarded, the probability of picking another defective pen is:

\[
\frac{17}{199}
\]

For the third pen:

\[
\frac{16}{198}
\]

Thus, the total probability is:

\[
\left( \frac{18}{200} \right) \left( \frac{17}{199} \right) \left( \frac{16}{198} \right) = 0.000621 \times 100\% = 0.0621\%
\]

\section*{Problem 3}

The following table displays the breakdown of attendees at an International Biology conference by country and their role in the company they were representing.

\begin{tabular}{|c|c|c|c|c|c|}
\hline
& Canada & France & South Korea & UK  & US  \\
\hline
CEO      & 138    & 45     & 4           & 19  & 117 \\
Director & 8      & 4      & 25          & 6   & 63  \\
Partner  & 23     & 7      & 3           & 20  & 103 \\
Chairman & 12     & 9      & 3           & 9   & 62  \\
Other    & 112    & 146    & 154         & 143 & 2103 \\
\hline
\end{tabular}

A random attendee is selected for an interview.

\textbf{a)} What is the probability that a Partner is selected, given that the attendee is from South Korea?\\
\textbf{Solution:} \\
The conditional probability is:

\[
P(\text{Partner} | \text{South Korea}) = \frac{3}{189} = \frac{1}{63}
\]

\textbf{b)} What is the probability that a Canadian is selected, given that the attendee is a director of the company?\\
\textbf{Solution:} \\
The conditional probability is:

\[
P(\text{Canadian} | \text{Director}) = \frac{8}{106}
\]

\textbf{c)} What is the probability that a director is selected, given that the attendee is Canadian?\\
\textbf{Solution:} \\
The conditional probability is:

\[
P(\text{Director} | \text{Canadian}) = \frac{8}{293}
\]

\textbf{d)} What is the probability that a CEO is selected, given that the attendee is from the continent of North America?\\
\textbf{Solution:} \\
The conditional probability is:

\[
P(\text{CEO} | \text{North America}) = \frac{255}{2741}
\]

\section*{Problem 4}

Every 6 months, university email requires that a new 5-digit password be set up. No digits are allowed to be repeated and it must be different from your last two passwords. If you let your computer randomly choose a 5-digit code for you with no repeating digits, what is the probability that it will choose one of the last 2 passwords you’ve had? Round your answer to five decimal places.

\textbf{Solution:} \\
The total number of possible passwords is:

\[
10 \times 9 \times 8 \times 7 \times 6 = 30240
\]

The probability that 2 of the computer-generated passwords match your last 2 is:

\[
\frac{2}{30240} = 0.00007
\]

\section*{Problem 5}

Virginia’s Veggie Café offers 5 types of homemade bread, 10 toppings, and 6 different condiments. How many different super sandwiches can be made if a super sandwich consists of 6 different toppings and 2 different condiments?

\textbf{Solution:} \\
The total number of combinations is:

\[
5 \times \binom{10}{6} \times \binom{6}{2} = 5 \times 210 \times 15 = 15750
\]

\section*{Problem 6}

Ashley’s Internet service is terribly unreliable. On any given day, there is a 15\% chance that her Internet connection will be lost. What is the probability that her Internet service is not broken for five days in a row?

\textbf{Solution:} \\
The probability that the Internet works on a given day is:

\[
P(\text{Works}) = 1 - 0.15 = 0.85
\]

The probability that it works for five days in a row is:

\[
(0.85)^5
\]

\section*{Problem 7}

Because Tristan has diabetes, he must ensure his daily diet includes 2 vegetables, 3 fruits, and 2 breads. At the grocery store, he has a choice of 20 vegetables, 8 fruits, and 5 breads.

\textbf{a)} In how many ways can he make up his daily requirements if he doesn’t eat 2 helpings of the same thing?\\
\textbf{Solution:} \\
The number of combinations is:

\[
\binom{20}{2} \times \binom{8}{3} \times \binom{5}{2} = 106400
\]

\textbf{b)} What’s the probability that a random choice from his possibilities would yield either carrots or spinach?\\
\textbf{Solution:} \\
The number of ways to include carrots or spinach is:

\[
P(A) + P(B) - P(A \cap B) = 19 + 19 - 1 = 37
\]

The probability is:

\[
\frac{20720}{106400} \approx 0.1947 \times 100\% = 19.47\%
\]
\end{document}